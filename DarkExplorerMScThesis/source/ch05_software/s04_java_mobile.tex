\section{Platforma mobilna (Java ME)}
\label{sec:javame-app}
Jednym z elementów rozwoju możliwości robota mobilnego Dark Explorer było
stworzenie aplikacji zarządzającej na urządzenia przenosne. W ramach tego zadania
stworzono aplikację działającą na platformie Java ME\footnote{Java ME - Java
Micro Edition}. Platforma ta jest dedykowana dla aplikacji tworzonych na
urządzenia mobilne o bardzo ograniczonych zasobach, takich jak telefony
komórkowe.

W związku z niską wydajnością procesorów oraz małą ilością pamięci w telefonach
komórkowych Java ME posiada ograniczony w stosunku do Java SE\footnote{Java SE -
Java Standard Edition, wersja platformy Java na komputery stacjonarne} zbiór klas
nazywanych konfiguracją. W Java ME wyróżniamy twa typy konfiguracji:
CDC\footnote{CDC -- Connected Device Configuration} dla urządzeń o lepszych
parametrach (smartphone'y) oraz CLDC\footnote{CLDC -- Connected Limited Device
Configuration} dla urządzeń o słabych parametrach (proste telefony komórkowe). W
tej pracy wykorzystana została konfiguracja CLDC.

Konfiguracje Java ME są uzupełniane przez profile MIDP\footnote{MIDP -- Mobile
Information Device Profile} które dodają swoje własne klasy do klas istniejących
w konfiguracji. Klasy te zapewniają wykonywanie odpowiednich zadań na konkretnych
elementach urządzenia mobilnego. Aplikacje wykorzystujące MIDP nazywane są
MIDletami i są uruchamiane w środowisku KVM\footnote{KVM - K Virtual Machine}.

K Virtual Machine jest niczym innym jak wirtualną maszyną Java opracowaną dla
konfiguracji CLDC. Jest ona bardzo ograniczona przez co posiada mniejsze
wymagania sprzętowe w porównaniu ze swoimi odpowiednikami z komputerów klasy PC.
Każdy producent urządzeń mobilnych musi zadbać o własną implementację maszyny
wirtualnej na której będą uruchamiane MIDlety.
\subsection{Narzędzia programistyczne}

\subsection{Aplikacja mobilna}