W ramach niniejszego dodatku zostanie przedstawiona szczegółowa dokumentacja
protokołu komunikacyjnego zaprojektowanego na potrzeby sterowania robotem Dark
Explorer przy użyciu technologii bluetooth. Zasady przesyłania
poleceń wraz z ogólnym zarysem sposobu podziału komunikatów został szczegółowo
omówiony w ramach rozdziału \ref{sec:bt-comm}. W tej części zaprezentowane
zostaną wszystkie dostępne polecenia sterujące wraz z~przykładami
formatów odpowiedzi w przypadku prawidłowego wykonania polecenia.

\section{Komunikaty sterujące}
W tej części zamieszczone zostały wszystkie komunikaty pozwalające w
sposób bezpośredni sterować zachowaniem robota oraz przełączać go pomiędzy
różnymi trybami pracy. Każde dostępne polecenie oraz powiązany z nim
komunikat został zobrazowane za pomocą graficznej reprezentacji jego logicznej
struktury. Dodatkowo każdy element struktury komunikatu został opatrzony
testowym komentarzem dokładnie wyjaśniającym jego funkcje. W przypadkach gdy jest to
konieczne zmieszczono również komunikat format komunikatu zwrotnego.

\subsection{Sterowanie silnikami}
\begin{figure}[h!]
 \centering
 \includegraphics[width=\textwidth]{../images/appendix/cmd_0x01.png}
 \caption{Schemat ramki odpowiedzialnej za sterowanie silnikami} 
 \label{fig:CMD_0x01}
\end{figure}

\begin{basedescript}{\desclabelstyle{\pushlabel}\desclabelwidth{25mm}}
\setlength{\parsep}{0pt}
\setlength{\itemsep}{0mm}
\setlength{\parskip}{0pt}
\item[ACK]
	Wartość 0x1 jeśli komunikat wymaga potwierdzenia
\item[EXT] 
	Wartość 0x1 jeśli komunikat zawiera dane o mocy silników
\item[RFU] 
	Bity zarezerwowane do przyszłego użycia
\item[RE MODE] 
	Tryb pracy silników prawych (0x0 - stop, 0x1 - wprost, 0x2 - wstecz)
\item[LE MODE] 
	Tryb pracy silników lewych (0x0 - stop, 0x1 - wprost, 0x2 - wstecz)
\item[SPEED] 
	Moc silników. Wartość z zakresu [0 - 254]
\end{basedescript}

\subsection{Sterowanie serwomechanizmem}
\begin{figure}[h!] 
 \centering
 \includegraphics[width=\textwidth]{../images/appendix/cmd_0x02.png}
 \caption{Schemat komunikatu odpowiedzialnego za sterowanie serwomechanizmem}
 \label{fig:CMD_0x02}
\end{figure}

\begin{basedescript}{\desclabelstyle{\pushlabel}\desclabelwidth{25mm}}
\setlength{\parsep}{0pt}
\setlength{\itemsep}{0mm}
\setlength{\parskip}{0pt}
\item[ACK]
	Wartość 0x1 jeśli komunikat wymaga potwierdzenia
\item[EXT] 
	Wartość 0x1 jeśli komunikat zawiera dane o położeniu
\item[RFU] 
	Bity zarezerwowane do przyszłego użycia
\item[LED] 
	Wartość 0x1 jeśli status diody LED ma zostać zmieniony na przeciwny
\item[SERVO] 
	Wartość 0x1 jeśli stan serwomechanizm ma zostać zmieniony na przeciwny
\item[POSITION] 
	Pozycja (kąt) pod jakim ma zostać ustawiona wieżyczka z kamerą.
\end{basedescript}

\subsection{Akwizycja obrazu z kamery}
\begin{figure}[h!] 
 \centering
 \includegraphics[width=\textwidth]{../images/appendix/cmd_0x03.png}
 \caption{Schemat komunikatu odpowiedzialnego za akwizycję obrazu z kamery}
 \label{fig:CMD_0x03}
\end{figure}

\begin{basedescript}{\desclabelstyle{\pushlabel}\desclabelwidth{25mm}}
\setlength{\parsep}{0pt}
\setlength{\itemsep}{0mm}
\setlength{\parskip}{0pt}
\item[ACK]
	Wartość 0x1 jeśli komunikat wymaga potwierdzenia
\item[EXT] 
	Wartość 0x1 jeśli komunikat zawiera dane o położeniu
\item[RFU] 
	Bity zarezerwowane do przyszłego użycia
\item[TYPE] 
	Typ operacji (0x0 - akwizycja obrazu, 0x1 transmisja pobranego obrazu)
\item[MODE] 
	Tryb pracy kamery (0x0 - odcienie szarości, 0x1 - kolor)
\item[RES] 
	Identyfikator rozmiaru obrazu według następującej specyfikacji:
	\begin{desc}
	\item[0x0] - 160x120
	\item[0x1] - 320x240
	\item[0x2] - 640x480
	\end{desc} 
\item[PART NO] 
	Numer fragmentu obrazu do pobrania w przypadku włączonego trybu transmisji
	obrazu
\end{basedescript}

\subsubsection{Format transmisji danych obrazu}
Ze względu na duży rozmiar danych do przesłania klient musi wysyłać
żądania pobrania każdego z elementów oddzielnie. Do zadań klienta należy
pobranie wszystkich elementów obrazu i złożenie ich w odpowiedniej kolejności.
Zdjęcie przesyłane jest w postaci kolejnych wierszy obrazu pobranego z kamery.
Do zakodowania każdego z pikseli obrazu wykorzystany został model kolorów YCbCr. 

\subsection{Sterowanie czujnikami}
\begin{figure}[h!] 
 \centering
 \includegraphics[width=\textwidth]{../images/appendix/cmd_0x04.png}
 \caption{Schemat komunikatu odpowiedzialnego za odpytywanie czujników}
 \label{fig:CMD_0x04}
\end{figure}

\begin{basedescript}{\desclabelstyle{\pushlabel}\desclabelwidth{25mm}}
\setlength{\parsep}{0pt}
\setlength{\itemsep}{0mm}
\setlength{\parskip}{0pt}
\item[ACK]
	Wartość 0x1 jeśli komunikat wymaga potwierdzenia
\item[EXT] 
	Wartość 0x1 jeśli komunikat zawiera dane rozszerzone
\item[RFU] 
	Bity zarezerwowane do przyszłego użycia
\item[CAL] 
	Ustawienie wartości 0x1 spowoduje uruchomienie procedury kalibrującej
	dla sensora podanego w sekcji SENSOR
\item[SENSOR] 
	Unikalny identyfikator typu pomiaru (czujnika) który ma zostać przeprowadzony,
	według następującej specyfikacji:
	\begin{desc}
	\item[0x0] - pomiar napięcia na baterii
	\item[0x1] - pomiar temperatury
	\item[0x2] - pomiar przyspieszenia statycznego
	\item[0x3] - informacje o odległości od najbliższej przeszkody
	\item[0x4] - informacje o położeniu na podstawie magnetometru
	\item[0x5] - informacje o kącie obrotu na podstawie
	żyroskopu\footnote{Pierwsze żądanie powoduje wyzerowanie stanu żyroskopu, drugie spowoduje pobranie kąta obrotu zarejestrowanego przez czujnik}
	\end{desc}
\end{basedescript}

\subsubsection{Format odpowiedzi dla poszczególnych czujników}
W zależności od wybranego typu czujnika znaczenie oraz typ przesyłanych w
odpowiedzi danych może być różny. Dlatego też informacja o formacie danych
została zróżnicowana ze względu na rodzaj pomiaru jaki czujnik przeprowadza. 
\begin{basedescript}{\desclabelstyle{\pushlabel}\desclabelwidth{25mm}}
\setlength{\parsep}{0pt}
\setlength{\itemsep}{0mm}
\setlength{\parskip}{0pt}
\item[Bateria]
	Na pierwszych 2 bajtach znajduje się wartość napięcia odczytanego na baterii
	(unsigned short), na kolejnym bajcie znajduje się wartość procentowa
	reprezentująca poziom naładowania baterii. 
\item[Czujnik temperatury]
	Wartość w zakresie [-127, 127] reprezentująca temperaturę w
	stopniach Celsjusza.
\item[Akcelerometr] 
	Kolejne pary bajtów zawierają wartość napięcia (unsigned short) odczytane na
	kolejnych osiach X, Y, Z czujnika przyspieszenia w wyniku działania przyspieszenia statycznego. 
\item[Czujniki odległości] 
	Na pierwszych 4 bajtach znajduje się napięcie odczytane kolejno na lewym i
	prawym czujniku odległości (unsigned short). Kolejne 2 bajty zawierają
	odległość od przeszkody odpowiednio dla lewego i prawego czujnika wyrażoną w
	centymetrach (unsigned char).
\item[Magnetometr]
	Kolejne pary bajtów zawierają wartość (short) składowych X oraz Y wektora
	indukcji pola magnetycznego. 
\item[Żyroskop]
	Kolejne pary bajtów zawierają wartość kąta (short) o jaki robot został
	obrócony odpowiednio wokół osi X, Y oraz Z. 
\end{basedescript}
\newpage
\subsection{Obsługa nagrywania i rekonstrukcji ścieżki powrotnej}
\begin{figure}[h!]
 \centering
 \includegraphics[width=\textwidth]{../images/appendix/cmd_0x05.png}
 \caption{Schemat ramki odpowiedzialnej za obsługę nagrywania i rekonstrukcji ścieżki powrotnej}
 \label{fig:CMD_0x05}
\end{figure}

\begin{basedescript}{\desclabelstyle{\pushlabel}\desclabelwidth{25mm}}
\setlength{\parsep}{0pt}
\setlength{\itemsep}{0mm}
\setlength{\parskip}{0pt}
\item[ACK]
	Wartość 0x1 jeśli komunikat wymaga potwierdzenia
\item[EXT] 
	Wartość 0x1 jeśli komunikat zawiera dane rozszerzone
\item[RFU] 
	Bity zarezerwowane do przyszłego użycia
\item[CMD] 
	Unikalny identyfikator akcji która ma zostać wykonana w związku z obsługą
	algorytmu rekonstrukcji ścieżki, według następującej specyfikacji:
	\begin{desc}
	\item[0x0] - rozpoczęcie nagrywania ścieżki
	\item[0x1] - zakończenie nagrywania ścieżki
	\item[0x2] - usunięcie danych ścieżki
	\item[0x3] - rozpoczęcie odtwarzania ścieżki
	\item[0x4] - zakończenie (przerwanie) odtwarzania ścieżki
	\item[0x5] - przesłanie do klienta danych ścieżki
	\end{desc}
\end{basedescript}

\subsection{Obsługa trybu autonomicznego}
\begin{figure}[h!]
 \centering
 \includegraphics[width=\textwidth]{../images/appendix/cmd_0x06.png}
 \caption{Schemat ramki odpowiedzialnej za obsługę trybu autonomicznego}
 \label{fig:CMD_0x06}
\end{figure}

\begin{basedescript}{\desclabelstyle{\pushlabel}\desclabelwidth{25mm}}
\setlength{\parsep}{0pt}
\setlength{\itemsep}{0mm}
\setlength{\parskip}{0pt}
\item[ACK]
	Wartość 0x1 jeśli komunikat wymaga potwierdzenia
\item[EXT] 
	Wartość 0x1 jeśli komunikat zawiera dane rozszerzone
\item[RFU] 
	Bity zarezerwowane do przyszłego użycia
\item[CMD] 
	Unikalny identyfikator akcji która ma zostać wykonana w związku z obsługą
	trybu autonomicznego, według następującej specyfikacji:
	\begin{desc}
	\item[0x0] - wyłączenie trybu autonomicznego
	\item[0x1] - włączenie trybu autonomicznego dla kształtów
	\item[0x2] - włączenie trybu autonomicznego dla twarzy
	\item[0x3] - pobranie współrzędnych zlokalizowanego obiektu
	\item[0x4] - pobranie binarnej wersji obrazu 
	\end{desc}
\end{basedescript}

\subsubsection{Format komunikatów zwrotnych}
Pobieranie zbianryzowanej wersji obrazu powoduje przesłanie strumienia 0 i 1
będących binarną reprezentacją analizowanego obrazu. Obraz przesyłany jest w
postaci strumienia, więc aplikacja klienta nie musi wysyłać żądań dla kolejnych
części obrazu.

Współrzędne zlokalizowanego obiektu są przesyłane w postaci 2 par punktów
pozwalających na jednoznaczne wyznaczenie prostokąta (bounding box) wokół
zlokalizowanego na obrazie obiektu.

\section{Komunikaty specjalne}
W tej części zamieszczone zostały komunikaty nie wpływające bezpośrednio na
zachowanie robota, ale mające za zadanie przekazanie informacji o aktualnym
stanie robota. Znajdują się tutaj między innymi komunikaty informujące o
błędach powstałych w czasie wykonania poleceń, jak również takie które pozwalają
sprawdzić aktualny stan połączenia oraz robota.

\subsection{Komunikat powitalny}
Celem komunikatu jest sprawdzenie poprawności nawiązanego połączenia. Polecenie
może zostać również wykorzystane do podtrzymania aktywności połączenia bluetooth
lub sprawdzenia poprawności działania robota po wystąpieniu krytycznego błędu.

\begin{figure}[h!]
 \centering
 \includegraphics[width=\textwidth]{../images/appendix/cmd_0x00.png}
 \caption{Schemat ramki komunikatu powitalnego (kontroli połączenia)} 
 \label{fig:CMD_0x00}
\end{figure}

\begin{basedescript}{\desclabelstyle{\pushlabel}\desclabelwidth{25mm}}
\setlength{\parsep}{0pt}
\setlength{\itemsep}{0mm}
\setlength{\parskip}{0pt}
\item[ACK]
	Wartość 0x1 jeśli komunikat wymaga potwierdzenia
\item[EXT] 
	Wartość 0x1 jeśli komunikat zawiera dane o mocy silników
\item[RFU] 
	Bity zarezerwowane do przyszłego użycia
\item[PATTERN] 
	Losowa sekwencja bitów która ma zostać odesłana przez robota w celu
	potwierdzenia poprawności połączenia i gotowości robota do dalszej pracy.
\end{basedescript}

\subsubsection{Definicja odpowiedzi}
Prawidłową odpowiedzią na komunikat powitalny powinna być ramka z polem STATE
ustawionym na wartość 0 i EOT na 1. W polu SIZE ramki odpowiedzi powinien
znaleźć się wzorzec który został przesłany do robota w ramach żądania.
Wszystkie nieprawidłowości w stosunku do tej definicji powinny być uznawane za
błędy. Polecenie zwrotne powinno zostać odesłane natychmiastowo, w przypadku
przedłużania się czasu odpowiedzi należy uznać to za błąd komunikacji. 
\newpage
\subsection{Komunikaty błędów}
W przypadku wystąpienia błędów ogólnych w działaniu robota lub w działaniu
poszczególnych funkcji robot może informować o wystąpieniu błędu za pomocą
specjalnego komunikatu. Pakiet zawierający informację o błędzie w swoim nagłówku
będzie zawierał identyfikator polecenia które wywołało błąd lub 0x0 w przypadku
błędów ogólnych. Dodatkowo w polu STATE i EOT będzie znajdować się 1. W polu
SIZE natomiast zamieszczony zostanie unikalny kod błędu identyfikujący
problem lub przyczynę powstania błędu.