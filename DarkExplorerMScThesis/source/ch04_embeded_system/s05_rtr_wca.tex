\section{Algorytm rekonstrukcji ścieżki powrotnej}
\label{sec:rtrwca}
Robot Dark Explorer został wyposażony w czujniki, które mają na celu dostarczenie
informacji niezbędnych do ustalenia toru ruchu robota. Niniejszy podrozdział
opisuje algorytm wykorzystujący dane z czujników w celu wyznaczenia trasy od
obecnego położenia do miejsca z którego robot został przyniesiony przez
operatora.

Algorytm rekonstrukcji ścieżki powrotnej składa się z dwóch części. Pierwsza z
nich odpowiedzialna jest za zapamiętywanie toru ruchu robota, druga natomiast za
odtworzenie ścieżki na podstawie zgromadzonych danych. Cały algorytm opiera się
na informacjach uzyskanych z dwóch czujników: żyroskopu oraz akcelerometru.
Czujnik przyspieszenia został zastosowany w celu określenia odległości jaką
przebył robot niesiony przez operatora. Natomiast żyroskop pozwala na uzyskanie
informacji o zmianie kierunku.

\subsection{Rejestracja trasy}
Aby możliwe stało się zrealizowanie implementacji algorytmu rekonstrukcji
ścieżki powrotnej konieczne jest w pierwszej kolejności zapamiętanie wszystkich
punktów charakterystycznych trasy po której poruszał się operator.

Pierwotnym podejściem do rozwiązania problemu wykrycia toru ruchu ciała
przenoszonego przez operatora było wyznaczanie przemieszczenia na
podstawie informacji o zmianach przyspieszenia uzyskanej z akcelerometru.
W celu obliczenia drogi po jakiej poruszało się ciało, mając jedynie informacje 
o przyspieszeniu, konieczne jest wykonanie dwukrotnego całkowania wartości 
otrzymanych z czujnika. Niestety operacja ta wymaga przeprowadzania pomiarów w
bardzo krótkich odstępach czasu w celu zminimalizowania błędów całkowania.
Nie mniej istotna jest także czułość i dokładność samego czujnika
przyspieszenia. Po wykonaniu wstępnych testów tego rozwiązania stwierdzono iż
otrzymywane rezultaty nie są zadowalające  i nie pozwalają na stworzenie
stabilnego rozwiązania w oparciu o opisywane podejście. 

Rozwiązaniem pozwalającym na uzyskanie satysfakcjonujących rezultatów okazała
się metoda wykrywania ilości kroków które wykonał operator robota podczas 
przemieszczania go w inne miejsce. Użytkownik korzystając z aplikacji
sterującej rozpoczyna proces nagrywania. Podczas wykonywania kroków, operator 
robota wykonuje mimowolne ruchy ręką w górę i w dół które są rejestrowane przez
czujniki robota. Dzięki wykrywaniu odpowiedniej sekwencji przyspieszeń jesteśmy
w stanie określić czy operator wykonał krok. Po wykryciu każdego kolejnego kroku
zapisywana jest wartość kąta pomiędzy kierunkiem ruchu z poprzedniego i obecnego
kroku. Prezentowane podejście pozwala na zapisanie całej trasy w postaci
sekwencji zmian kierunku następujących po przemieszczeniu się robota o odległość
jednego kroku. Rozwiązanie to pozwala zarejestrowanie wszystkich punktów
charakterystycznych trasy przy jednoczesnym ograniczeniu ilości danych
potrzebnych do stworzenia jej logicznej reprezentacji.

Algorytm wykrywania kroków zrealizowany został w oparciu o zainstalowany w
robocie akcelerometr trójosiowych. Szczegółowy opis zasady działania
akcelerometru oraz sposobu implementacji procedury zliczania kroków zamieszczony
został w rozdziale \ref{sec:accelometer}. Informacje na temat kąta obrotu jaki
wykonał użytkownik podczas nagrywania trasy mogą być pobierane z jednego z dwóch
czujników. Domyślnym czujnikiem jest żyroskop, który pozwala z niezwykłą
dokładnością wykonywać pomiary prędkości kątowej, a co za tym idzie umożliwia
precyzyjne wyznaczenie kąta o jaki dokonany został obrót. Więcej informacji na
temat zasady działania żyroskopu można znaleźć w rozdziale \ref{sec:gyro}.
Sensorem zapasowym jest magnetometr. Pozwala on wyznaczyć obecną orientację 
urządzenia względem bieguna magnetycznego ziemi. Robot jest w stanie na 
podstawie danych otrzymywanych z kompasu jednoznacznie wyznaczyć kąt o jaki 
został obrócony. Z zasadami działania magnetometru oraz
napotkanymi problemami które dyskryminują ten czujnik jako podstawowy przy
rejestrowaniu i odtwarzaniu toru ruchu można zapoznać się w rozdziale
\ref{sec:mag}.

\subsection{Rekonstrukcja trasy}
Zakończenie nagrywanie trasy użytkownik sygnalizuje za pomocą odpowiedniego
komunikatu wysłanego za pośrednictwem dowolnej aplikacji sterującej. Robot
poinformuje użytkownika o zakończeniu nagrywania ścieżki wyświetlając na ekranie
LCD informacje na temat liczby kroków zapamiętanych w trakcie ostatniej sesji.
Następnie, operator, ustawia robota na podłożu, przodem, w kierunku którym robot
będzie miał wracać. Rozpoczęcie procedury odtwarzania zapamiętanego toru ruchu
rozpocznie się po wysłaniu polecenia inicjalizującego algorytm rekonstrukcji
ścieżki powrotnej. Robot korzystając z informacji dostarczanych na bieżąco z
żyroskopu oraz danych zapisanych w postaci nagranej ścieżki będzie starał się
powrócić do miejsca z którego wystartował użytkownik. Dodatkowo, jeżeli na
zapamiętanej ścieżce pojawi się przeszkoda uniemożliwiająca odtworzenie ścieżki
zostanie ona wykryta za pomocą dołączonych czujników odległości i rozpocznie się
procedura mająca na celu uniknięcie kolizji z przeszkodą. W chwili gdy robot
ominie przeszkodę będzie starał się, jeżeli to możliwe, powrócić na oryginalną
ścieżkę ruchu. Procedura ta ma na celu zagwarantowanie, że robot prawidłowo
dotrze do miejsca z którego wyruszył. W przypadku gdy robot stwierdzi, że powrót
na oryginalną ścieżkę nie jest możliwy lub ominięcie przeszkody nie jest możliwe
procedura rekonstrukcji zostanie automatycznie zakończona. 
