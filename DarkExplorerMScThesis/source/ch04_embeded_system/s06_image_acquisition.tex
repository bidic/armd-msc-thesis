\section{Modernizacja sposobu pobierania obrazu z kamery} 
\label{sec:img-acq}
W celu zwiekszenia możliwości robota Dark Eksplorer w dziedzinie przetwarzania obrazu niezbędne było polepszenie rozdzielczości zdjęć otrzymywanych z wbudowanej kamery. W wersji bazowej maksymalne rozdzielczości z jakimi było możliwe pobieranie obrazów to 160x100 pikseli w kolorze oraz 320x200 pikseli w odcieniach szarości. Wartości te zostały narzucone przez ograniczenia pamięci podręcznej mikrokontrolera ARM który posiada jedynie 64kB szybkiej pamięci SRAM. Ilość ta wystarczyla na pobranie maksymalnie $320*200=64000$ bajtów danych pozostawiając 1536 bajtów na zmienne niezbędne do poprawnego działania oprogramowania systemu wbudowanego.

Wykorzystywana kamera posiada osiem wyjść równoległych odpowiadających za jeden bajt danych. Wszystkie osiem bitów na wyjściu zmienia się z każdym taktem zegara sterującego kamery, podanego na jej wejście. W taki sposób z każdym cyklem  zegara, kamera oddaje do naszej dyspozycji dane z kolejnej porcji obrazu. W konfiguracji początkowej zegar wejściowy kamery był tworzony w sposób czysto programowy. Jedno z wyjść GPIO mikrokontrolera było ustawiane na przemian raz w stan wysoki, a raz w stan niski. Niewątpliwie podejście to znacząco ułatwia synchronizację pomiędzy sygnałem zegarowym a momentem pobierania danych eliminując możliwość zczytania danych z wyjścia kamery w momencie w którym wyjścia te są w stanie nieustalonym.

\begin{figure}[ht!]
 \centering \includegraphics[height=85mm]{../images/ch04/dataflash_structure.png}
 \caption{Schemat struktury logicznej AT45DB321B wraz z zazaznaczonymi operacjami zapisu\cite{AT45DB321BApplicationNote}.}
 \label{fig:DataFlashStruct}
\end{figure}

Przedstawiony problem niedostatecznej ilości pamięci został rozwiązany poprzez wykorzystanie pamięci DataFlash wbudowanej w moduł mikrokontrolera. Użyty układ AT45DB321B\cite{AT45DB321BDataSheet} dostarcza 32 megabitów pamięci którą możemy zarządzać poprzez interfejs szeregowy SPI\footnote{SPI -- Serial Peripheral Interface Bus}. Zastosowana pamięć posiada dwa bufory po 528 bajtów pojemności. Strutkura logiczna AT45DB321B wraz z zaznaczonymi operacjami zapisu została przedstawiona na rysunku \ref{fig:DataFlashStruct}.  Podczas przenoszenia danych z jednego bufora do pamięci DataFlash możliwy jest zapis informacji do drugiego bufora. W ten sposób emulują one zapis danych do pamięci DataFlash w trybie ciągłym. Zarys algorytmu ciągłego zapisu danych do AT45DB321B przedstawia schemat na rysunku \ref{fig:DataFlashConstantWrite}

\begin{figure}[ht!]
 \centering \includegraphics[height=70mm]{../images/ch04/dataflash_constant_write.png}
 \caption{Schemat procedury ciągłego zapisu do układu AT45DB321B\cite{AT45DB321BApplicationNote}.}
 \label{fig:DataFlashConstantWrite}
\end{figure}

Czas zapisu danych na strone pamięci układu AT45DB321B nie jest stały. Wymusza to zmiane podejścia do zegara sterującego kamerą z koncepcji programowej na sprzętową tak aby okres zegara był jednorodny. Niejednorodny zegar spowodowałby przebarwienia na obrazie wynikające z zasady działania mechanizmu dobierania ekspozycji wbudowanego w kamerze PO6030K. W celu uzyskania sprzętowego zegara użyte zostało urządzenie peryferyjne mikrokontrolera AT91SAM7S generującego sygnał prostokątny o określonej częstotliwości. Z powodu multipleksowania tego urządzenia z jednym z pinów GPIO mikrokontrolera podłączonego do wyjścia danych z kamery, konieczne było zastosowanie przeplotu w taśmie podłączeniowej kamery w celu dostarczenia odpowiednich sygnałów do prawidłowych wejść/wyjść.

Pojawiły się również problemy podczas uruchamiania interfejsu szeregowego SPI za pomocą którego mikrokontroler komunikuje się z układem AT45DB321B. Okazało się bowiem, że wyjścia/wejścia odpowiedzialne za obsługę tego interfejsu są multipleksowane z kontrolerem PWM\footnote{PWM -- Pulse Width Modulation} odpowiedzialnym za odpowiednie wysterowanie silników robota. W konfiguracji podstawowej silniki te były kontrolowane przez 4 niezależne sygnały PWM pozwalające na obracanie się każdego koła robota z inną prędkością. Stwierdzono, iż taka funkcjonalność nie jest niezbędna i wykorzystano trzy zbędne wyjścia sygnałów PWM do obsługi interfejsu SPI, a przy pomocy własnoręcznie wykonanej zworki dostarczono jeden sygnał PWM to wszystkich czterech silników.

Wszystkie te operacje pozwoliły na odbiór obrazu o maksymalnych rozdzielczościach oferowanych przez układ PO6030K to znaczy 640x480 pikseli w odcieniach szarości oraz 640x480 pikseli w kolorze.