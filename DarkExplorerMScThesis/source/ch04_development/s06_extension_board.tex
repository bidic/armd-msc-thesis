\section{Płyta rozszerzeń}
\label{ch:ExpanderChapter}
Wszystkie elementy elektroniczne stworzone na rzecz tej pracy magisterskiej
zostały połączone przy pomocy płyty rozszerzeń. Doprowadza ona zasilanie oraz
odpowiedni interfejs komunikacyjny do poszczególnych urządzeń. Płyta ta została
przygotowana przy pomocy oprogramowania Eagle\footnote{Eagle -- Easily Applicable
Graphical Layout Editor} w wersji edukacyjnej. Niestety z powodu ograniczeń na
maksymalny rozmiar płytki stworzonej za pomocą tego oprogramowania z licencja
edukacyjną, konieczne było podzielenie płyty rozszerzeń na dwie części.

\begin{figure}[!ht]
 \centering
 \includegraphics[height=75mm]{../images/ch04/extension_board-sch.png}
 \caption{Schemat części cyfrowej płyty rozszerzeń}
 \label{fig:ExtBoardSch}
\end{figure}

\begin{figure}[!ht]
 \centering
 \includegraphics[height=75mm]{../images/ch04/extension_board.png}
 \caption{Layout cyfrowej części płyty rozszerzeń}
 \label{fig:ExtBoardPCB}
\end{figure}

Jedna część płyty rozszerzeń (rys. \ref{fig:ExtBoardSch}) jest odpowiedzialna za
wszystkie elementy cyfrowe które komunikują się z robotem przy pomocy interfejsu
$I^{2}C$. Druga natomiast (rys. \ref{fig:AdcMultiplexerSch} pozwala na
podłączanie elementów analogowych których sygnały są interpretowane przez
przetwornik analogowo cyfrowy będący jednym z urządzeń peryferyjnych
mikrokontrolera ARM.

\begin{figure}[!ht]
 \centering
 \includegraphics[height=50mm]{../images/ch04/adcmultiplexer-sch.png}
 \caption{Schemat części analogowej płyty rozszerzeń}
 \label{fig:AdcMultiplexerSch}
\end{figure}

\begin{figure}[!ht]
 \centering
 \includegraphics[height=50mm]{../images/ch04/adcmultiplexer-brd.png}
 \caption{Layout analogowej części płyty rozszerzeń}
 \label{fig:AdcMultiplexerPCB}
\end{figure}