\section{Inercjalny system nawigacyjny}
W celu rozwinięcia możliwości robota, zamontowano w nim elementy przy pomocy
których, została podjęta próba stworzenia inercjalnego systemu nawigacyjnego
(INS\footnote{Inertial Navigation System}). Założeniem było, aby robot mobilny
był w stanie zapamiętać tor ruchu po jakim się porusza, gdy jest niesiony na ręce
osoby operującej nim. Następnie na podstawie wykonanych pomiarów robot miał
powrócić po zapamiętanym torze w miejsce początkowe. Elementy użyte do wykonania
INS zostały wybrane pod kątem walorów ekonomicznych. Nie były przeprowadzane
testy porównawcze pomiędzy podzespołami danego typu. Poniższy rozdział omawia
pokrótce czym jest INS, opisuje zasadę działania jego elementów oraz sposób
wykorzystania tych podzespołów do osiągnięcia wystarczająco dobrych efektów.

\subsection{Wprowadzenie do INS}
% http://citeseerx.ist.psu.edu/viewdoc/download?doi=10.1.1.63.7402&rep=rep1&type=pdf

Inercjalny system nawigacyjny jest to narzędzie służące do określenia położenia,
prędkości oraz orientacji obiektu w przestrzeni, bez korzystania z żadnych
zewnętrznych elementów naprowadzających, które byłyby dla niego punktem
odniesienia. Wykorzystuje on jedynie elementy wbudowane, składające się na
inercjalną jednostkę pomiarową (IMU\footnote{Inertial Measurement Unit}).
Inercjalne systemy nawigacyjne mają zastosowanie tam, gdzie jest potrzebna
informacja o aktualnym położeniu obiektów, natomiast nie ma możliwości odbioru
sygnału zewnętrznego, wymaganego do działania na przykład urządzeń GPS. Systemy
tego typu są stosowane w: samolotach, statkach, łodziach podwodnych, pojazdach
bezzałogowych czy statkach kosmicznych. Rozwój miniaturowych układów
elektromechanicznych MEMS\footnote{MEMS ?? } otworzył przed nami cały
wachlarz potencjalnych nowych zastosowań INS, np. do śledzenia ruchów ludzi bądź
zwierząt.

System nawigacyjny o którym mowa w tym rozdziale oblicza swoje położenie na
podstawie ciągłego badania przyspieszenia liniowego oraz prędkości kątowej. INS
musi otrzymać na starcie wartości początkowe położenia oraz prędkości z jaką się
porusza, aby móc zacząć wyznaczać dalsze przemieszczenie i zmiany w orientacji.

Robot mobilny został wyposażony w IMU składające się z następujących elementów
MEMS: żyroskopu oraz akcelerometru trójosiowego, a także magnetometru
dwuosiowego. Zasada działania poszczególnych elementów oraz opis ich
wykorzystania można znaleźć w kolejnych podrozdziałach.
