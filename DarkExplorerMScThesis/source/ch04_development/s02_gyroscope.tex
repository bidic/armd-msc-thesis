\subsection{Żyroskop}
Do budowy inercjalnej jednostki pomiarowej został użyty żyroskop trójosiowy wykonany w technologii MEMS. 
Żyroskopy wyprodukowane w tej technologii są wykorzystywane np.: podczas stabilizacji obrazu w aparatach 
cyfrowych lub kontrolerach do gier wideo. Ich główną zaletą jest mały rozmiar. W poniższym podrozdziale 
zostanie  przedstawiona zasada działania takiego żyroskopu oraz sposób wykorzystania go w robocie mobilnym.

\subsubsection{Zasada działania}
%http://www.electroiq.com/index/display/nanotech-article-display/4659348781/articles/small-times/nanotechmems/mems/sensors/2010/11/introduction-to-mems-gyroscopes.html
Żyroskopy MEMS korzystają z pozornej siły Coriolis'a do pomiaru prędkości kątowej z jaką obraca się ciało.
\\
\textcolor{red}{TODO: TUTAJ RYSUNEK}
\\
Załóżmy że ciało (A) o masie (m) porusza się z prędkością (v) oraz układ w którym przemieszcza się to ciało obraca sie 
z prędkością kątową (omega). W takich warunkach ciało o którym mowa zostanie odchylone od kierunku przemieszczania się 
wyznaczonego przez wektor prędkości (v). Odchylenie to będzie spowodowane siłą Coriolisa (F) którą można opisać przy 
pomocy wzoru (wzor). Żyroskopy MEMS wykorzystują to zjawisko określając przemieszczenie drgającego ciała, które jest
defakto jedną z okładek kondensatora, jako zmianę pojemności.
\\
Żyroskopy tego typu posiadają dwa oscylujące miniaturowe elementy które poruszają się w przeciwnych względem
siebie kierunkach (\textcolor{red}{TODO: NR RYSUNKU}). 
\\
\textcolor{red}{TODO: TUTAJ RYSUNEK}
\\
Jeżeli urządzenie do którego przymocowany jest żyroskop zacznie się obracać, spowoduje to
wychylenie się oscylujących elementów w przeciwnych kierunkach. Wychylenie to z koleji powoduje zmianę pojemności, a
różnica pomiędzy pojemnościami zmierzonymi przy pomocy obydwu ruchomych elementów jest proporcjonalna do prędkości
kątowej. W ten sposób zmierzona szybkość obrotu jest następnie reprezentowana jako wynik analogowy (napięcie) lub cyfrowy
(wartość liczbowa).
\\
Wynik działania żyroskopu MEMS jest odporny na zmianę przyspieszenia liniowego, w szczególności na przyspieszenie ziemskie.
Przyspieszenie liniowe wywoła wychylenie się jednoczesnie obydwu elementów oscylujących w tym samym kierunku. W ten sposób
nie będzie wykryta żadna różnica pojemności. Prędkość kątowa wskazywana przez żyroskop nie ulegnie zmianie. Dzięki temu
żyroskopy MEMS są odporne na wstrząsy, uderzenia oraz wibracje.
\\
Podstawowe wielkości określające żyroskop:
\begin{itemize}
 \item zakres [dps\footnote{dps - stopień na sekundę}] -- definiuje wartość maksymalną i minimalną jaką żyroskop jest w stanie
 zmierzyć. Często żyroskop ma kilka zakresów z których możemy wybrać jeden odpowiadający naszym potrzebą.
 \item czułość [mdps/digit] -- wielkość określająca wartość minimalną jaką żyroskop może zmierzyć
 \item zmiana czułości względem temperatury [\%] -- określa jak bardzo pomiary urządzenia są podatnę na zmianę temperatury
 \item zakres poziomu zero [dps] -- określa zakres w jakim pomiary mogą się wachać w chwili gdy żyroskop pozostaje w spoczynku
 \item zmiana zakresu poziomu zero względem temperatury [dps/ STOPIEŃ C] -- definiuje zmiany pomiarów żyroskopu pozostającego w spoczynku
 względem temperatury
 \item nieliniowość [\% FS] -- parametr określa maksylany procentowe odchylenie wartości na wyjściu żyroskopu od dopasowanej do nich linii prostej
 \item przepustowość [Hz] -- definiuje częstotliwość z jaką są wykonywane pomiary
\end{itemize}
\textcolor{red}{TODO: POPRAWIĆ ZAPIS JEDNOSTEK}

\subsubsection{Kalibracja i odczytywanie wyników}
Żyroskopy są zazwyczaj fabrycznie testowane i kalibrowane pod kątem zakresów pomiarów poziomu zerowego oraz czułości. 
Jednak umieszczeniu elementu ma plytce PCB zostaje on poddany naprężeniom przez co może być potrzebna dodatkowa kalibracja układu.

Wartości wyjściowe żyroskopu można przedstawić za pomocą równania:

\textcolor{red}{TODO: WZÓR i zapis jednostek} Rt = SC x (Rm - R0) 

Gdzie
 Rt (dps): rzeczywista prędkość kątowa
 Rm  (LSBs): pomiar z żyroskopu
 R0 (LSBs): wartość zerowa reprezentująca brak ruchu
 SC (dps/LSB): czułość

W celu kompensacji niestabilności żyroskopu należy pozostawić urządzenie nieruchome i wykonać ok 1000 pomiarów. Następnie
konieczne jest obliczenie wartości średniej z wykonanych pomiarów. W ten sposób określimy średnie wychylenie od wartości
zerowej żyroskopu, czyli nasze R0.

Podczas używania żyroskopu do pomiaru bardzo małych kątów należy jeszcze wziąć pod uwagę minimalne odchylenia wartości w
zależności od zmiany temperatury.

Prędkość kątowa opisujemy wzorem: 
\textcolor{red}{TODO: WZÓR i zapis jednostek}
omega = dfi / dt

Stąd wzór na kąt o jaki zostało obrócone ciało poruszające się z predkością kątową !!!OMEGA!!! w czasie !!!T!!!.
\textcolor{red}{TODO: WZÓR i zapis jednostek}
Fi = całka z omega po dt

Do obliczenia kąta można wykorzystać jedną z numerycznych metod całkowania, np. metodę trapezów.
Metoda ta polega na podzieleniu pola pod wykresem całkowanej funkcji na wąskie trapezy, a następnie
zsumowanie pól tych trapezów. \textcolor{red}{TODO: NR RYSUNKU}
\textcolor{red}{TODO: RYSUNEK}

Po wykorzystaniu metody trapezów, wzór na kąt o jaki żyroskop został obrócony w czasie pomiędzy dwoma pomiarami, 
można zapisać następująco:
\textcolor{red}{TODO: WZÓR i zapis jednostek}
Fi = CF * (dt)(omega(ti-1) + omega(ti))/2
Gdzie
 CF - współczynnik korygujący
 dt - czas jaki upłynął pomiędzy dwoma pomiarami
 omega(ti-1) - szybkość kątowa odczytana z poprzedniego pomiaru
 omega(ti) - szybkość kątowa odczytana z obecnego pomiaru

W taki sposób otrzymujemy wzór na kąt całkowity pomiędzy orientacją początkowa a orientacją końcową ciala.
\textcolor{red}{TODO: WZÓR i zapis jednostek}
Fic = Fi0 + Suma od 1 do n Fi(i) 

Do otrzymania jak najlepszych wyników konieczne jest określenie współczynnika korygującego pomiar żyroskopu.
W tym celu porównujemy wyniki pomiarów żyroskopu z pomiarami innych przyrządów np. magnetometru wykorzystując wzór:
\textcolor{red}{TODO: WZÓR i zapis jednostek}
CF = Fio/Fig
Gdzie
 CF - współczynnik korygujący
 Fio - kąt obliczony przy pomocy urządzenia zewnętrznego
 Fig - kąt wyznaczony przez żyroskop

\subsubsection{Opis zastosowanego elementu i budowy modułu}
