\documentclass[a4paper,12pt, oneside]{mwbk}

\usepackage[utf8]{inputenc}
\usepackage[T1]{polski}
\usepackage{savesym}
\savesymbol{itemize*}
\savesymbol{enumerate*}
\usepackage{mdwlist}
\restoresymbol{MDW}{itemize*}
\restoresymbol{MDW}{enumerate*}
\usepackage{helvet}
\usepackage{graphicx}
\usepackage{color}
\usepackage{geometry}
\usepackage{cite}
\usepackage{url}
\usepackage{subfig}
\usepackage{setspace}
\usepackage{parskip}
\usepackage{indentfirst}
\usepackage{float}
\usepackage{listings}
\usepackage{wrapfig}
\usepackage{enumitem}
\usepackage{newclude}
\usepackage[table]{xcolor}
\usepackage{eurosym}
\setlength{\parskip}{2ex}
\setlength{\parindent}{30pt}

%\geometry{hmargin={2cm, 2cm}, height=10.0in}
\geometry{total={210mm,297mm}, left=20mm,right=20mm,bindingoffset=10mm,
top=25mm, bottom=25mm} 

\let\stdsection\section
\renewcommand\section{\newpage\stdsection}

%\includeonly{introduction}

% \makeatletter
% \renewcommand\paragraph{\@startsection{paragraph}{4}{\z@}%
%   {-3.25ex\@plus -1ex \@minus -.2ex}%
%   {1.5ex \@plus .2ex}%
%   {\normalfont\normalsize\bfseries}}
% \makeatother

\sloppy %zakaz wydłużania linii (gdy nie może złożyć) 
\brokenpenalty=10000 %nie dieli wyrazów pomiędzy stronami 
\clubpenalty=10000 %nie pozostawia sierot pojedynczych
\widowpenalty=10000 %nie pozostawia wdów pojedynczych

\newenvironment{desc}
{\begin{description}
  \setlength{\itemsep}{0mm}
  \setlength{\parskip}{0pt}
  %\setlength{\labelwidth}{15mm}
  \setlength{\labelsep}{10mm}
  %\setlength{\itemindent}{15mm}
  %\setlength{\listparindent}{\parindent}
  \setlength{\parsep}{1pt}}
{\end{description}}

\newenvironment{item-list}
{\begin{itemize}
  \setlength{\itemsep}{0mm}
  \setlength{\parskip}{0pt}
  %\setlength{\labelwidth}{15mm}
  %\setlength{\labelsep}{10mm}
  %\setlength{\itemindent}{5mm}
  %\setlength{\listparindent}{\parindent}
  \setlength{\parsep}{1pt}}
{\end{itemize}}

\begin{document}

\thispagestyle{empty}
%% ------------------------ NAGLOWEK STRONY ---------------------------------
\includegraphics[height=37.5mm]{../images/agh_nzw_a_pl_1w_wbr_rgb_150ppi.jpg}\\
\rule{30mm}{0pt}
{\large \textsf{Wydział Fizyki i Informatyki Stosowanej}}\\
\rule{\textwidth}{3pt}\\
\rule[2ex]
{\textwidth}{1pt}\\
\vspace{7ex}
\begin{center}
{\LARGE \bf \textsf{Praca magisterska}}\\
\vspace{13ex}
% --------------------------- IMIE I NAZWISKO -------------------------------
{\bf \Large \textsf{Łukasz Hanusiak, Mariusz Nowacki}}\\
\vspace{3ex}
{\sf\small kierunek studiów:} {\bf\small \textsf{informatyka stosowana}}\\
\vspace{1.5ex}
{\sf\small specjalność:} {\bf\small \textsf{informatyka w nauce i technice}}\\
\vspace{10ex}
%% ------------------------ TYTUL PRACY --------------------------------------
{\bf \huge \textsf{Rozwój systemu robota mobilnego}}\\
\vspace{14ex}
%% ------------------------ OPIEKUN PRACY ------------------------------------
{\Large Opiekun: \bf \textsf{dr hab. inż. Marek Idzik}}\\
\vspace{22ex}
{\large \bf \textsf{Kraków, kwiecień 2011}}
\end{center}
%% =====  STRONA TYTUŁOWA PRACY MAGISTERSKIEJKIEJ ====

\newpage

%% =====  TYŁ STRONY TYTUŁOWEJ PRACY MAGISTERSKIEJKIEJ ====
{\sf Oświadczam, świadomy(-a) odpowiedzialności karnej za poświadczenie nieprawdy, że niniejszą pracę dyplomową wykonałem(-am) osobiście i samodzielnie i  nie korzystałem(-am) ze źródeł innych niż wymienione w pracy.}

\vspace{14ex}

\begin{center}
\begin{tabular}{lr}
~~~~~~~~~~~~~~~~~~~~~~~~~~~~~~~~~~~~~~~~~~~~~~~~~~~~~~~~~~~~~~~~~ &
................................................................. \\
~ & {\sf (czytelny podpis)}\\
\end{tabular}
\end{center}

\vspace{14ex}

\begin{center}
\begin{tabular}{lr}
~~~~~~~~~~~~~~~~~~~~~~~~~~~~~~~~~~~~~~~~~~~~~~~~~~~~~~~~~~~~~~~~~ &
................................................................. \\
~ & {\sf (czytelny podpis)}\\
\end{tabular}
\end{center}

%% =====  TYL STRONY TYTULOWEJ PRACY MAGISTERSKIEJKIEJ ====

\newpage
\rightline{Kraków, ?? kwiecień 2011}
\begin{center}
{\bf Tematyka pracy magisterskiej i praktyki dyplomowej
Łukasza Hanusiaka oraz Mariusza Nowackiego
studentów V roku studiów kierunku informatyka stosowana, informatyka w nauce i technice}\\
\end{center}

Temat pracy magisterskiej:
{\bf Rozwój systemu robota mobilnego}\\

\begin{tabular}{rl}

Opiekun pracy:                  & dr hab. inż. Marek Idzik\\
Recenzenci pracy:               & ??? ??? \\
Miejsce praktyki dyplomowej:    & WFiIS AGH, Kraków\\
\end{tabular}

\begin{center}
{\bf Program pracy magisterskiej i praktyki dyplomowej}
\end{center}

\begin{enumerate}
\item Omówienie realizacji pracy magisterskiej z opiekunem.
\item Zebranie i opracowanie literatury dotyczącej tematu pracy.
\item Praktyka dyplomowa:
\begin{itemize}
\item zapoznanie się z ideą...,
\item uczestnictwo w eksperymentach/przygotwanie oprogramowania...,
\item dyskusja i analiza wyników...
\item sporządzenie sprawozdania z praktyki.
\end{itemize}
\item Kontynuacja obliczeń związanych z tematem pracy magisterskiej.
\item Zebranie i opracowanie wyników obliczeń.
\item Analiza wyników obliczeń numerycznych, ich omówienie i zatwierdzenie przez opiekuna.
\item Opracowanie redakcyjne pracy.
\end{enumerate}

\noindent
Termin oddania w dziekanacie: ?? kwiecień 2011\\[1cm]

\begin{center}
\begin{tabular}{lcr}
.............................................................. & ~~~ &
.............................................................. \\
(podpis kierownika katedry) & & (podpis opiekuna) \\
\end{tabular}
\end{center}

\newpage

\noindent
Recenzja
\newpage
Recenzja

\linespread{1.3}
\selectfont

\newpage
Ze względu na obszerny zakres działań niniejsza praca magisterska była wykonywana przez zespół dwuosobowy. W celu ułatwienia zrozumienia treści zawartej w pracy nie rozdzielono jej na dwie części pisane przez poszczególnych autorów lecz została podzielona rozdziałami pod względem logicznym. Przypisanie opracowań redakcyjnych elementów pracy magisterskiej do autorów przedstawia poniższa lista: 

\textbf{Łukasz Hanusiak opracował:}
\begin{item-list}
  \item Historia i rozwój robotyki (rozdział \ref{ch:history}).
  \item Akclelerometr trzyosiowy (podrozdział \ref{sec:accelometer}).
  \item Czujnik odległości (podrozdział \ref{sec:ir-sensors}).
  \item Rozbudowa obudowy robota (podrozdział \ref{sec:casing}).
  \item Narzędzia do rozwoju systemu wbudowanego dla systemu Windows
  (podrozdział \ref{sec:embeded-win-tools}).
  \item Protokół komunikacji bluetooth (podrozdział \ref{sec:bt-comm}).
  \item Lokalizacja twarzy na obrazie (podrozdział \ref{sec:face-detect}).
  \item Opis biblioteki do zarządzania robotem dla platformy .NET (podrozdział
  \ref{subsec:sdk-.net}).
  \item Platforma mobilna - Windows Mobile (podrozdział \ref{sec:wm-app}).
  \item Specyfikacja poleceń protokołu komunikacji (dodatek \ref{ch:bt-spec})
\end{item-list}

\textbf{Mariusz Nowacki opracował:}
\begin{item-list}
  \item Analiza bazowej konfiguracji robota (rozdział \ref{ch:analysis}).
  \item Inercjalny system nawigacyjny (podrozdział \ref{sec:ins}).
  \item Żyroskop (podrozdział \ref{sec:gyro}).
  \item Magnetometr (podrozdział \ref{sec:mag}).
  \item Wyświetlacz LCD (podrozdział \ref{sec:lcd}).
  \item Płyta rozszerzeń (podrozdział \ref{ch:ExpanderChapter}).
  \item Narzędzia do rozwoju systemu wbudowanego dla systemu Linux (podrozdział
  \ref{sec:embeded-linux-tools}).
  \item Opis biblioteki AT91LIB (podrozdział \ref{sec:at91lib}).
  \item Modernizacja sposobu pobierania obrazu z kamery (podrozdział
  \ref{sec:img-acq}).
  \item Opis biblioteki do zarządzania robotem dla języka Java (podrozdział
  \ref{subsec:sdk-java}).
  \item Aplikacja do sterowania robotem dla komputerów PC (Java) (podrozdział
  \ref{sec:java-app}).
  \item Platforma mobilna - Java ME (podrozdział \ref{sec:javame-app}).
  \item Wykonanie płytek drukowanych układów elektronicznych (dodatek \ref{ch:boards-design})
\end{item-list}

\textbf{Wspólnie opracowano:}
\begin{item-list}
  \item Wprowadzenie
  \item Algorytm rekonstrukcji ścieżki powrotnej (podrozdział \ref{sec:rtrwca}).
  \item Podsumowanie
\end{item-list}

\newpage
\tableofcontents

\newpage 
\chapter*{Wprowadzenie}
\include*{ch01_introduction/s01_introduction}
\include*{ch01_introduction/s02_history_of_robotics}

\newpage
\chapter{Analiza bazowej konfiguracji robota}
\label{ch:analysis}
\include*{ch02_analysis/s01_hardware}
\include*{ch02_analysis/s02_software}

\newpage
\chapter{Rozwój sprzętowej warstwy robota mobilnego}
\include*{ch03_development/s00_ins}
\include*{ch03_development/s01_accelerometer}
\include*{ch03_development/s02_gyroscope}
\include*{ch03_development/s03_magnetometer}
\include*{ch03_development/s04_ir_sensors}
\include*{ch03_development/s05_lcd_display}
\include*{ch03_development/s06_extension_board}
\include*{ch03_development/s07_casing}

\newpage
\chapter{Rozwój oprogramowania systemu wbudowanego}
\include*{ch04_embeded_system/s01_windows}
\include*{ch04_embeded_system/s02_linux}
\include*{ch04_embeded_system/s03_at91lib}
\include*{ch04_embeded_system/s04_communication}
\include*{ch04_embeded_system/s05_rtr_wca}
\include*{ch04_embeded_system/s06_image_acquisition}
\include*{ch04_embeded_system/s07_face_detection}

\newpage
\chapter{Aplikacje zarządzające robotem dla PC i urządzeń mobilnych}
\include*{ch05_software/s01_libraries}
\include*{ch05_software/s02_java}
\include*{ch05_software/s03_windows_mobile}
\include*{ch05_software/s04_java_mobile}

\newpage
\chapter*{Podsumowanie}
\include*{ch06_summary/s01_summary}

\newpage 
\appendix
\chapter{Specyfikacja poleceń protokołu komunikacji}
\label{ch:bt-spec}
\include*{appendix/btp_spec}
\chapter{Wykonanie płytek drukowanych układów elektronicznych}
\label{ch:boards-design}
\include*{appendix/pcb}
\chapter{Plik konfiguracyjny programatora: triton.cfg}
\include*{appendix/triton-cfg}
\chapter*{Kod źródłowy}
%\begin{lstlisting} \end{lstlisting}
%\lstinputlisting[language=Python]{source_filename.py}
\listoffigures
\listoftables

\newpage

\bibliography{bibliography}{}
\bibliographystyle{plainpl}

\newpage

\end{document}