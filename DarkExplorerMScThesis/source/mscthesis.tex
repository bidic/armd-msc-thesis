\documentclass[a4paper,12pt, oneside]{mwbk}

\usepackage[utf8]{inputenc}
\usepackage[T1]{polski}
\usepackage{savesym}
\savesymbol{itemize*}
\savesymbol{enumerate*}
\usepackage{mdwlist}
\restoresymbol{MDW}{itemize*}
\restoresymbol{MDW}{enumerate*}
\usepackage{helvet}
\usepackage{graphicx}
\usepackage{color}
\usepackage{geometry}
\usepackage{cite}
\usepackage{url}
\usepackage{setspace}
\usepackage{parskip}
\usepackage{indentfirst}
\usepackage{float}
\usepackage{listings}
\usepackage{wrapfig}
\usepackage{enumitem}
\usepackage{newclude}
\usepackage[table]{xcolor}
\setlength{\parskip}{2ex}
\setlength{\parindent}{30pt}

%\geometry{hmargin={2cm, 2cm}, height=10.0in}
\geometry{total={210mm,297mm}, left=20mm,right=20mm,bindingoffset=10mm,
top=25mm, bottom=25mm} 

%\includeonly{introduction}

% \makeatletter
% \renewcommand\paragraph{\@startsection{paragraph}{4}{\z@}%
%   {-3.25ex\@plus -1ex \@minus -.2ex}%
%   {1.5ex \@plus .2ex}%
%   {\normalfont\normalsize\bfseries}}
% \makeatother

\sloppy %zakaz wydłużania linii (gdy nie może złożyć) 
\brokenpenalty=10000 %nie dieli wyrazów pomiędzy stronami 
\clubpenalty=10000 %nie pozostawia sierot pojedynczych
\widowpenalty=10000 %nie pozostawia wdów pojedynczych

\newenvironment{desc}
{\begin{description}
  \setlength{\itemsep}{0mm}
  \setlength{\parskip}{0pt}
  %\setlength{\labelwidth}{15mm}
  \setlength{\labelsep}{10mm}
  %\setlength{\itemindent}{15mm}
  %\setlength{\listparindent}{\parindent}
  \setlength{\parsep}{1pt}}
{\end{description}}

\begin{document}

\thispagestyle{empty}
%% ------------------------ NAGLOWEK STRONY ---------------------------------
\includegraphics[height=37.5mm]{../images/agh_nzw_a_pl_1w_wbr_rgb_150ppi.jpg}\\
\rule{30mm}{0pt}
{\large \textsf{Wydział Fizyki i Informatyki Stosowanej}}\\
\rule{\textwidth}{3pt}\\
\rule[2ex]
{\textwidth}{1pt}\\
\vspace{7ex}
\begin{center}
{\LARGE \bf \textsf{Praca magisterska}}\\
\vspace{13ex}
% --------------------------- IMIE I NAZWISKO -------------------------------
{\bf \Large \textsf{Łukasz Hanusiak, Mariusz Nowacki}}\\
\vspace{3ex}
{\sf\small kierunek studiów:} {\bf\small \textsf{informatyka stosowana}}\\
\vspace{1.5ex}
{\sf\small specjalność:} {\bf\small \textsf{informatyka w nauce i technice}}\\
\vspace{10ex}
%% ------------------------ TYTUL PRACY --------------------------------------
{\bf \huge \textsf{Rozwój platformy robota mobilnego}}\\
\vspace{14ex}
%% ------------------------ OPIEKUN PRACY ------------------------------------
{\Large Opiekun: \bf \textsf{dr hab. inż. Marek Idzik}}\\
\vspace{22ex}
{\large \bf \textsf{Kraków, kwiecień 2011}}
\end{center}
%% =====  STRONA TYTUŁOWA PRACY MAGISTERSKIEJKIEJ ====

\newpage

%% =====  TYŁ STRONY TYTUŁOWEJ PRACY MAGISTERSKIEJKIEJ ====
{\sf Oświadczam, świadomy(-a) odpowiedzialności karnej za poświadczenie nieprawdy, że niniejszą pracę dyplomową wykonałem(-am) osobiście i samodzielnie i  nie korzystałem(-am) ze źródeł innych niż wymienione w pracy.}

\vspace{14ex}

\begin{center}
\begin{tabular}{lr}
~~~~~~~~~~~~~~~~~~~~~~~~~~~~~~~~~~~~~~~~~~~~~~~~~~~~~~~~~~~~~~~~~ &
................................................................. \\
~ & {\sf (czytelny podpis)}\\
\end{tabular}
\end{center}

%% =====  TYL STRONY TYTULOWEJ PRACY MAGISTERSKIEJKIEJ ====

\newpage
\rightline{Kraków, ?? kwiecień 2011}
\begin{center}
{\bf Tematyka pracy magisterskiej i praktyki dyplomowej
Łukasza Hanusiaka oraz Mariusza Nowackiego
studentów V roku studiów kierunku informatyka stosowana, informatyka w nauce i technice}\\
\end{center}

Temat pracy magisterskiej:
{\bf Rozwój platformy robota mobilnego}\\

\begin{tabular}{rl}

Opiekun pracy:                  & dr hab. inż. Marek Idzik\\
Recenzenci pracy:               & ??? ??? \\
Miejsce praktyki dyplomowej:    & WFiIS AGH, Kraków\\
\end{tabular}

\begin{center}
{\bf Program pracy magisterskiej i praktyki dyplomowej}
\end{center}

\begin{enumerate}
\item Omówienie realizacji pracy magisterskiej z opiekunem.
\item Zebranie i opracowanie literatury dotyczącej tematu pracy.
\item Praktyka dyplomowa:
\begin{itemize}
\item zapoznanie się z ideą...,
\item uczestnictwo w eksperymentach/przygotwanie oprogramowania...,
\item dyskusja i analiza wyników...
\item sporządzenie sprawozdania z praktyki.
\end{itemize}
\item Kontynuacja obliczeń związanych z tematem pracy magisterskiej.
\item Zebranie i opracowanie wyników obliczeń.
\item Analiza wyników obliczeń numerycznych, ich omówienie i zatwierdzenie przez opiekuna.
\item Opracowanie redakcyjne pracy.
\end{enumerate}

\noindent
Termin oddania w dziekanacie: ?? kwiecień 2011\\[1cm]

\begin{center}
\begin{tabular}{lcr}
.............................................................. & ~~~ &
.............................................................. \\
(podpis kierownika katedry) & & (podpis opiekuna) \\
\end{tabular}
\end{center}

\newpage

\noindent
Na kolejnych dwóch stronach proszę dołączyć kolejno recenzje pracy popełnione przez Opiekuna oraz Recenzenta (wydrukowane z systemu MISIO i podpisane przez odpowiednio Opiekuna i Recenzenta pracy). Papierową wersję pracy (zawierającą podpisane recenzje) proszę złożyć w dziekanacie celem rejestracji co najmniej na tydzień przed planowaną obroną.

\linespread{1.3}
\selectfont

\newpage
\tableofcontents

\newpage 
\chapter*{Wprowadzenie}
\include*{ch01_introduction/s01_introduction}
\include*{ch01_introduction/s02_history_of_robotics}

\newpage
\chapter{Analiza bazowej konfiguracji robota}
\include*{ch02_analysis/s01_hardware}
\include*{ch02_analysis/s02_software}

\newpage
\chapter{Rozwój platformy robota mobilnego -- środowisko rozwojowe systemu wbudowanego}
\begin{figure}[!ht]
 \centering
 \includegraphics[height=150mm]{../images/ch03/darkexplorer_platform.png}
 \caption{Struktura platformy robota mobilnego po zakończeniu prac}
 \label{fig:DarkExplorerPlatform}
\end{figure}
\include*{ch03_software/s01_windows}
\include*{ch03_software/s02_linux}

\newpage
\chapter{Rozwój platformy robota mobilnego -- środowisko rozwojowe dla komputerów stacjonarnych}
\include*{ch03_software/s03_java}
\include*{ch03_software/s04_c_sharp}

\newpage
\chapter{Rozwój platformy robota mobilnego -- środowisko rozwojowe dla urządzeń mobilnych}
\include*{ch03_software/s05_windows_mobile}
\include*{ch03_software/s06_java_mobile}

\newpage
\chapter{Rozwój platformy robota mobilnego -- warstwa sprzętowa}
\include*{ch04_development/s00_ins}
\include*{ch04_development/s01_accelerometer}
\include*{ch04_development/s02_gyroscope}
\include*{ch04_development/s03_magnetometer}
\include*{ch04_development/s04_ir_sensors}
\include*{ch04_development/s05_lcd_display}
\include*{ch04_development/s06_extension_board}
\include*{ch04_development/s07_casing}

\chapter{Rozwój platformy robota mobilnego -- firmware}
\include*{ch05_embeded_system/s01_communication}
\include*{ch05_embeded_system/s02_rtr_wca}
\include*{ch05_embeded_system/s03_face_detection}

\newpage
\chapter*{Podsumowanie}
\include*{ch06_summary/s01_summary}

\newpage 
\appendix
\chapter{Specyfikacja poleceń protokołu komunikacji}
\include*{appendix/btp_spec}
\chapter{Wytrawianie płytek układów elektronicznych}
\include*{appendix/pcb}
\chapter*{Kod źródłowy}
%\begin{lstlisting} \end{lstlisting}
%\lstinputlisting[language=Python]{source_filename.py}
\listoffigures
\listoftables

\newpage

\bibliography{bibliography}{}
\bibliographystyle{plainpl}

\newpage

\end{document}