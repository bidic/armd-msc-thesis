Celem pracy był rozwój platformy sprzętowej i programowej robota mobilnego wraz
z oprogramowaniem umożliwiającym zdalne sterowanie robotem. Charakter pracy
wymagał podzielenia procesu realizacji zadań na klika etapów z których najważniejsze
zamieszczone zostały poniżej.
\begin{itemize}
  \item zapoznanie się i analiza możliwości robota Dark Explorer,
  \item odszukanie narzędzi oraz przygotowanie środowiska do rozwoju systemu
  wbudowanego robota dla systemu Windows oraz Linux,
  \item wybór zestawu czujników dodatkowych pozwalających rozszerzyć
  funkcjonalność robota bez konieczności ingerencji w dotychczasowe rozwiązanie,
  \item wykonanie obudowy oraz elektroniki umożliwiających podłączenie czujników
  do wolnych portów robota, 
  \item stworzenie modułowego systemu wbudowanego umożliwiającego sterowanie
  robotem wraz z użyciem dodatkowych czujników
  \item zaprojektowanie i wykonanie bibliotek zewnętrznych umożliwiających
  swobodne tworzenie oprogramowania sterującego robotem dla urządzeń
  stacjonarnych i przenośnych,
  \item przygotowanie przykładowych aplikacji sterujących pozwalających na
  zaprezentowanie możliwości robota po rozbudowie
\end{itemize}
Rozbudowany w ramach pracy magisterskiej robot w pełni realizuje cele założone
przez temat pracy. Unowocześniona wersja nie tylko znacząco rozszerza
funkcjonalność swego poprzednika ale również dzięki poczynionym modyfikacjom w
oprogramowaniu i elektronice robota znacząco ułatwia dalszą jego rozbudowę.

Rozbudowana wersja robota została wyposażona w dodatkowe czujniki wśród których
znaleźć można: akcelerometr, żyroskop, magnetometr oraz dalmierze IR. Wszystkie
wspomniane sensory są wykorzystywane przez robota do rejestracji
oraz rekonstrukcji ścieżki powrotnej po której poruszał się operator trzymający
robota w ręku. Zaimplementowany algorytm odtwarzania ścieżki nie wymaga do swojego działania
żadnej zewnętrznej infrastruktury. Dodatkowym atutem jest fakt, iż robot
wykorzystuje czujniki nie tylko podczas nagrywania przebytej drogi, ale również
podczas jej odtwarzania. Gwarantuje to, że działanie algorytmu jest całkowicie
niezależne od wpływów środowiska zewnętrznego np. śliska powierzchnia. Mechanizm
rekonstrukcji ścieżki wymaga od użytkownika, aby po zakończeniu
nagrywania ścieżki robot był zwrócony w kierunku powrotnym. Mimo dołożonych
starań zaprojektowane rozwiązanie nie gwarantuje, że robot zawsze dotrze do
punktu startowego. Wynika to z braku informacji o długości kroku wykonanego
przez operatora. Długość kroku została przyjęta jako wartość stała wyznaczona w
sposób doświadczalny i nie zawsze musi odpowiadać stanowi faktycznemu.
Dodatkowym utrudnieniem są ruchy wykonywane przez użytkownika w
trakcie zapamiętywania trasy takie jak np. nieutrzymywanie czoła robota
równolegle do kierunku ruchu operatora. Powoduje to rejestrowanie błędnych
parametrów trasy, co skutkuje deformacją odtwarzanego toru ruchu robota podczas
powrotu. 

Rozbudowana została również warstwa oprogramownia robota. Zebrano i
usystematyzowano wiedzę i narzędzia potrzebne do tworzenia oprogramowania
uruchamianego na nowej wersji robota Dark Explorer. Dołożono wszelkich starań,
aby możliwe stało się tworzenie aplikacji klienckich nie tylko na urządzenia
stacjonarne, ale również mobilne, takie jak telefony komórkowe. Zaprojektowane
biblioteki programistyczne umożliwiają korzystanie z funkcji robota, bez
konieczności zagłębiania się w szczegóły jego działania. 

Podczas pracy nad robotem udało się pokonać szereg problemów związanych z
ograniczeniami narzuconymi przez pierwotną konfigurację sprzętową. Mała ilość
wejść oraz wyjść zarówno cyfrowych jak i analogowych, została roszerzona przy
pomocy odpowiednio ekspandera GPIO dla interfejsu $I^{2}C$ oraz multipleksera
analogowo--cyfrowego. Przy wykorzystaniu tej samej metody, możliwe jest
rozszerzanie wejść/wyjść robota w praktycznie nieograniczony sposób. Udało się
także otrzymać zdjęcia w maksymalnej rozdzielczości, oferowanej przez kamerę
wbudowaną w robocie. Umożliwia to wykorzystanie kamery do przeprowadzania
bardziej złożonej analizy obrazu, niż było to możliwe w przypadku poprzedniej
wersji robota. Podjęto próbę implementacji algorytmu wykrywającego twarz, który
jest uruchamiany bezpośrednio przez mikrokontroler wbudowany w robota. 

Kolejne etapy rozwoju robota mogłyby wiązać się z przyspieszeniem akwizycji oraz
transmisji obrazu z kamery. W przypadku transmisji konieczne byłoby
wykorzystanie innego interfejsu komunikacji z modułem bluetooth (np. USB) lub
całkowita wymiana tego modułu na inną technologie. Przyspieszenie akwizycji
obrazu może zostać zrealizowane za pomocą dedykowanego programowalnego układu
FPGA lub poprzez rozszerzenie dostępnej pamięci podręcznej mikrokontrolera.
Drugie podejście najprawdopodobniej będzie wiązało się z wymianą samego
mikrokontrolera. W celu polepszenia działania inercjalnego systemu
nawigacyjnego, istotne byłoby rozważenie wymiany obecnie używanego akcelerometru
na inny o lepszych parametrach oraz podjęcie próby wyznaczenia za jego pomocą
dystansu jaki przebył robot. Następnym elementem poprawiającym obecną
funkcjonalność robota, może być dołączenie do niego kolejnych czujników
odległości, co pozwoliłoby na lepsze omijanie przeszkód. 
