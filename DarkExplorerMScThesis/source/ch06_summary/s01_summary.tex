Celem pracy był rozwój platformy sprzętowej i programowej robota mobilnego wraz z
oprogramowaniem umożliwiającym zdalne sterowanie robotem. Charakter pracy wymagał
podzielenia procesu realizacji zadań na klika etapów z których najważniejsze
zamieszczone zostały poniżej.

\begin{itemize}
  \item Zapoznanie się z dotychczasowymi możliwościami robota Dark Explorer oraz
  analiza prawdopodobnych ścieżek rozwoju.
  \item Odszukanie niezbędnych narzędzi oraz przygotowanie kompletnego
  środowiska programistycznego dla systemu Windows oraz Linux
  które umożliwia rozwój oprogramowania sterującego pracą mikrokontrolera
  zainstalowanego w robocie.
  \item Wybór zestawu czujników dodatkowych pozwalających rozszerzyć 
  funkcjonalność robota bez konieczności ingerencji w jego konfigurację bazową.
  \item Wykonanie obudowy oraz elektroniki umożliwiającej szybkie podłączenie
  wszystkich czujników do wolnych portów robota. 
  \item Stworzenie modułowego systemu wbudowanego umożliwiającego sterowanie
  robotem wraz z użyciem dodatkowych czujników.
  \item Zaprojektowanie i wykonanie bibliotek zewnętrznych umożliwiających
  swobodne tworzenie oprogramowania sterującego robotem dla urządzeń
  stacjonarnych i przenośnych.
  \item Przygotowanie przykładowych aplikacji sterujących pozwalających na
  zaprezentowanie możliwości robota po zakończeniu prac.
\end{itemize}

Rozbudowany w ramach pracy magisterskiej robot w pełni realizuje cele założone
przez temat pracy. Unowocześniona wersja nie tylko znacząco rozszerza
funkcjonalność swego poprzednika ale również dzięki poczynionym modyfikacjom w
oprogramowaniu i~elektronice robota znacząco ułatwia jego dalszą rozbudowę.

Rozbudowana wersja robota została wyposażona w dodatkowe czujniki wśród których
znaleźć można: akcelerometr, żyroskop, magnetometr oraz dalmierze IR. Wszystkie
wspomniane sensory są wykorzystywane przez robota do rejestracji oraz
rekonstrukcji trasy po której poruszał się operator trzymający robota w ręku.
Zaimplementowany algorytm odtwarzania ścieżki nie wymaga do swojego działania
żadnej zewnętrznej infrastruktury. Dodatkowym atutem jest fakt, iż robot
wykorzystuje czujniki nie tylko podczas nagrywania przebytej drogi, ale również
podczas jej odtwarzania. Umożliwia to robotowi analizowanie parametrów środowiska
w którym się porusza w czasie rzeczywistym co~gwarantuje, że~robot będzie w
stanie ominąć przeszkody które pojawią się na jego drodze w czasie odtwarzania
zapamiętanej trasy. Mechanizm rekonstrukcji ścieżki wymaga od~użytkownika, aby po
zakończeniu nagrywania ścieżki robot był zwrócony w kierunku powrotnym. Mimo
dołożonych starań zaprojektowane rozwiązanie nie gwarantuje, że~robot zawsze
prawidłowo dotrze do punktu startowego. Wynika to z braku informacji o~długości
kroku wykonanego przez operatora która na potrzeby obecnego rozwiązania została
przyjęta jako wartość stała wyznaczona w sposób doświadczalny i nie zawsze musi
odpowiadać stanowi faktycznemu. Dodatkowym utrudnieniem są ruchy wykonywane przez
użytkownika w trakcie zapamiętywania trasy takie jak np. nieutrzymywanie czoła
robota równolegle do kierunku ruchu operatora. Powoduje to rejestrowanie błędnych
parametrów trasy, co~skutkuje deformacją odtwarzanego toru ruchu robota podczas
powrotu.

Rozbudowana została również warstwa oprogramowania robota. Zebrano i
usystematyzowano wiedzę i narzędzia potrzebne do tworzenia oprogramowania
uruchamianego na nowej wersji robota Dark Explorer. Zaprojektowano oraz
zrealizowano modułową wersję oprogramowania sterującego pracą
robota. Poprawia ona działanie dotychczasowych funkcji (np. komunikacja
bluetooth, obsługa kamery), ale umożliwia również korzystanie z~modułów
rozszerzeń i czujników dodanych w trakcie realizacji pracy magisterskiej.
Dołożono wszelkich starań, aby możliwe stało się tworzenie aplikacji klienckich nie tylko na
urządzenia stacjonarne, ale również mobilne do których zaliczyć można na
przykład współczesne telefony komórkowe czy tablety. Zaprojektowane
klienckie biblioteki programistyczne umożliwiają korzystanie z funkcji robota,
bez konieczności zagłębiania się w~szczegóły jego działania.

Podczas pracy nad robotem udało się pokonać szereg problemów związanych
z~ograniczeniami narzuconymi przez pierwotną konfigurację sprzętową i
programową. Mała ilość wejść oraz wyjść zarówno cyfrowych jak i analogowych, została rozszerzona przy
pomocy odpowiednio ekspandera GPIO dla interfejsu $I^{2}C$ oraz multipleksera
analogowo--cyfrowego. Przy wykorzystaniu tej samej metody, możliwe jest
rozszerzanie wejść/wyjść robota w praktycznie nieograniczony sposób. 

Udało się także otrzymać zdjęcia w maksymalnej rozdzielczości, oferowanej przez kamerę
wbudowaną w robocie. Umożliwia to wykorzystanie kamery do przeprowadzania
bardziej złożonej analizy obrazu, niż było to możliwe w przypadku poprzedniej
wersji robota. W ramach pracy podjęto próbę implementacji algorytmu
lokalizacji twarzy, który jest uruchamiany bezpośrednio przez mikrokontroler
wbudowany w robota co przy obecnej ilości dostępnych zasobów było niebagatelnym
wyzwaniem.

Kolejne etapy rozwoju robota mogłyby wiązać się z przyspieszeniem akwizycji oraz
transmisji obrazu z kamery. W przypadku transmisji konieczne byłoby
wykorzystanie innego interfejsu komunikacji z modułem bluetooth (np. USB) lub
całkowita wymiana tego modułu na technologię umożliwiającą
wydajniejszą transmisję danych. Przyspieszenie akwizycji obrazu może zostać
zrealizowane za pomocą dedykowanego programowalnego układu FPGA lub poprzez rozszerzenie dostępnej pamięci podręcznej mikrokontrolera.
Drugie podejście najprawdopodobniej będzie wiązało się z wymianą samego
mikrokontrolera, w takiej sytuacji należałoby rozważyć możliwość wymiany
mikrokontrolera na~szybszy co będzie miało kluczowe znaczenie podczas
przetwarzania danych z kamery. W~celu polepszenia działania inercjalnego systemu
nawigacyjnego, istotne byłoby rozważenie wymiany obecnie używanego akcelerometru
na model z lepszymi parametrami oraz podjęcie próby wyznaczenia za jego pomocą
dystansu jaki przebył robot. Można również rozważyć wykorzystanie
filtrów Kalmana do eliminowania niedoskonałości zastosowanych czujników
i metod uzyskiwania z nich danych. Następnym elementem poprawiającym obecną
funkcjonalność robota, może być dołączenie do niego kolejnych czujników
odległości, co~pozwoliłoby na lepsze omijanie przeszkód, a w przypadku
wykorzystania sonaru możliwe stałoby się tworzenie mapy pomieszczenia w którym
robot się porusza. 
