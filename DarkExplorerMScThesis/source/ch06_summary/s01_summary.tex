Celem pracy był rozwój platformy sprzętowej i programowej robota mobilnego wraz
z systemem umożliwiającym zdalne sterowanie robotem. Charakter pracy wymagał
podzielenia procesu realizacji pracy na klika etapów z których najważniejsze
zamieszczone zostały poniżej.
\begin{itemize}
  \item zapoznanie się i analiza możliwości robota Dark Explorer,
  \item odszukanie narzędzi oraz przygotowanie środowiska do rozwoju systemu
  wbudowanego robota dla systemu Windows oraz Linux,
  \item wybór zestawu czujników dodatkowych pozwalających rozszerzyć
  funkcjonalność robota bez konieczności ingerencji w dotychczasowe rozwiązanie,
  \item wykonanie obudowy oraz elektroniki umożliwiających podłączenie czujników
  do wolnych portów robota, 
  \item stworzenie modułowego systemu wbudowanego umożliwiającego sterowanie
  robotem wraz z użyciem dodatkowych czujników
  \item zaprojektowanie i wykonanie bibliotek zewnętrznych umożliwiających
  swobodne tworzenie oprogramowania sterującego robotem dla urządzeń
  stacjonarnych i przenośnych,
  \item przygotowanie przykładowych aplikacji sterujących pozwalających na
  zaprezentowanie możliwości robota po rozbudowie
\end{itemize}
Rozbudowany w ramach pracy magisterskiej robot w pełni realizuje cele założone
przez temat pracy. Unowocześniona wersja nie tylko znacząco rozszerza
funkcjonalność swego poprzednika ale również dzięki poczynionym modyfikacją w
oprogramowaniu i elektronice robota znacząco ułatwia dalszą rozbudowę robota. 
