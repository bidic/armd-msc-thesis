\section{Platforma mobilna (Windows Mobile 6.1)}

Mobilne urządzenia przenośne z dnia na dzień zyskują na popularności. Każdego
dnia spotykamy się z nimi w domu, w pracy czy spacerując po parku. Z całą pewnością
można stwierdzić, iż większa część społeczeństwa obecnie jest w posiadaniu
telefonu komórkowego, komputera przenośnego czy też jakiegoś innego urządzenia
mobilnego. Wszystkie z wspomnianych urządzeń posiadają dedykowaną platformę
software'ową. Do najlepiej znanych we współczesnym świecie zaliczyć można między
innymi: Windows Mobile, iPhone, BlackBerry, Symbian OS, Android, Maemo, OpenMoko
itp. Każda z wymienionych platform posiada inną genezę jak również swoje mocne i
słabe strony.

Platformy takie jak Windows Mobile, BlackBerry czy iPhone ograniczone są do
urządzeń dedykowanych docelowo do współpracy z wspomnianymi środowiskami. Obok
różnorakich problemów z jakimi zmagają się wspomniane wcześniej platformy do
jednego z najpoważniejszych zaliczyć można bardzo ograniczone w niektórych
aspektach API. Nawet tak przenośna platforma jak Java na urządzeniach przenośnych
nie zawsze się sprawdza ze względu na liczne braki oraz różnice w API zmuszające
programistów do tworzenia kodu dedykowanego dla konkretnego urządzenia. Symbian
oraz Windows Mobile wypadają na tym tle nieco lepiej ponieważ wspierają szerszą
gamę urządzeń jak również ich API daje więcej możliwości niż ma to miejsce na
przykład w przypadku Javy. Głównym powodem takiego stanu rzeczy jest bardzo
szeroki i różnorodny asortyment platform sprzętowych utrudniający stworzenie
jednolitej i w pełni wykorzystującej wszystkie możliwości urządzenia platformy
programistycznej. Dostępne w chwili obecnej OpenSource'owe i wieloplatformowe
rozwiązania znajdują się ciągle we wczesnej fazie rozwoju i nie są jeszcze
powszechnie znane przez środowiska twórców oprogramowania.

Firma Microsoft wypuściła po raz pierwszy na światło dzienne swoją platformę
mobilną w latach 90-tych. Natomiast w roku 2002 pojawiła się pierwsza platforma
Windows CE.NET. Zapoczątkowało to popularyzację urządzeń Pocket PC opartych o
system Windows CE 3.0 oraz późniejsze wersje. Dalszy rozwój bezprzewodowych
technologi telekomunikacyjnych pozwolił na integracje telefonu z komputerem
osobistym. Wspomniane urządzenia Pocket PC z 2002 roku wspierały między innymi
standard GSM, GPRS, bluetooth oraz umożliwiały użytkownikom dostęp do sieci
bezprzewodowych. W między czasie rozwojowi ulegały urządzenia typu SmartPhone
które koncepcyjnie były bardzo zbliżone do Pocket PC jednakże były one bardziej
zbliżone do telefonu niż komputera osobistego. Podstawową różnicą pomiędzy
Smartphone i Pocket PC jest fakt iż urządzenia Pocket PC posiadają ekran
dotykowy, a Smartphone wyposażone są jedynie w przyciski umożliwiające sterowanie
urządzeniem. Każde z tych urządzeń posiadało inny zestaw aplikacji pomocniczych
oraz wspierało inne standardy i technologie.

W chwili obecnej większość urządzeń Pocket PC oraz Smartphone działają w oparciu
o system Windows Mobile 5 oraz Windows Mobile 6. Nowoczesne urządzenia Pocket PC
wyposażone są w procesor o taktowaniu 500-600 MHz oraz od 64-128 MB pamięci RAM.
Najnowsze urządzenia z tej grupy wyposażane są w 1 GHz procesor oraz 512 MB
pamięci.
\subsubsection{Modele aplikacji}
Istnieje klika modeli rozwoju aplikacji dla Windows Mobile, a wybór docelowego
modelu został pozostawiony programiście. Pierwszy z nich służy do tworzenia
aplikacji w kodzie natywnym. Aplikacje pisane zgodnie z tym modelem cechują się
wysoką wydajnością, bezpośrednim dostępem do sprzęu oraz małym zużyciem zasobów.
Do rozwoju tego rodzaju aplikacji korzysta się z reguły ze środowiska do
rozwijania aplikacji z użyciem Embeded Visual C++. Główną wadą tego modelu jest
niska przenośność pomiędzy różnymi platformami zwłaszcza jeżeli aplikacja
korzysta z urządzeń specyficznych dla danego modelu urządzenia. Docelowo więc za
pomocą tego modelu tworzy się biblioteki i narzędzia ułatwiających tworzenie
bardziej skomplikowanych aplikacji. Jeżeli więc insteresuje nas tworzenie
wysokopoziomowych aplikacji z GUI, skierowanych bezpośrednio do użytkowników,
zaleca się tworzenie tego typu aplikacji za pomocą kodu zarządzalnego z użyciem
takich języków jak C\# czy Visual Basic. Rozwijanie aplikacji w oparciu o kod
zarządzalny pozwala na stworzenie programu który będzie mógł w pełni
wykorzystywać możliwości oferowane przez Microsoft .NET Compact Framework.
Umożliwia to programiście tworzenie rozproszonych systemów mobilnych pracujących
zarówno w modelu ze stałym połączeniem jak i bez. Sport aczęść narzędzi
dostępnych w ramach .NET Compact Framework jest również wykorzystywana do rozwoju
aplikacji na komputery stacjonarne. Biblioteka została zaprojektowana docelowo na
urządzenia o ograniczonych zasobach co w połączeniu z możliwościami języków z
rodziny .NET oraz integracją z Visual Studio daje nam profesjonalny zestaw
narzędzi do tworzenia aplikacji mobilnych.
\\
Trzecim modelem tworzenia programów pod Windows Mobile jest wykorzystywanie kodu
serwera do pracy z wieloma różnymi typami urządzeń poprzez jeden wspólny kod
bazowy i model cienkiego klienta. Oczywiście tego typu podejście ma sens jedynie
gdy możemy zagwarantować stabilny kanał komunikacyjny pomiędzy urządzeniem
klienta, a serwerem. Każdy z przedstawionych modeli idalnie sprawdza się jeżeli
tylko wprawidłowy sposób wybierzemy model najbardziej odpowiadający potrzebom
naszej aplikacji.

\subsubsection{Graficzny interfejs użytkownika}
Dzięki wygodnemu systemowi projektowania GUI
dostępnego w Visual Studio tworzenie interfejsu
użytkownika dla platformy mobilnej jest niemal tak proste jak w przypadku
tradycyjnych aplikacji. 

\subsection{Środowisko rozwojowe}
Tworzenie aplikacji działających na urządzeniach pod kontrolą systemu Windows
Mobile jest niemal tak samo proste jak tworzenie zwykłych aplikacji na komputery
stacjonarne. Niemniej jednak do stworzenia w pełni funkcjonalnego środowiska
rozwojowego konieczne jest przejście przez klika kroków przygotowawczych
związanych z instalacją potrzebnych aplikacji narzędziowych. \\
\\
Przed rozpoczęciem przygody z tworzeniem aplikacji dla systemu Windows Mobile
konieczne jest zainstalowanie Microsoft Visual Studio. Zaleca się aby Visual
Studio było w wersji 2005 lub 2008. Niestety narzędzia umożliwiające rozwijanie
aplikacji mobilnych nie są poprawnie wykrywane przez Visual Studio 2010 oraz
poprzednie wydania w wersji Express. Dlatego też koniecznością jest instalacja
środowiska w wersji Standard lub Professional. Każda z tych wersji może zostać
pobrana w wersji czasowej ze stron firmy Microsoft lub w wersji pełnej z MSDNAA.
Visual Studio posłuży nam nie tylko do edycji kodu aplikacji ale pozwoli również
w prosty sposób budować, debugować oraz przygotować instalator finalnej wersji
aplikacji. Po poprawnym zainstalowaniu środowiska rozwojowego konieczne jest
pobranie i zainstalowanie dostępnych paczek serwisowych dostępnych dla wybranej
wersji Visual Studio. Pozwoli to uniknąć nieprzyjemnych niespodzianek podczas
instalacji bibliotek narzędziowych i późniejszej pracy.\\
\\
Jeżeli posiadamy już zainstalowaną kopię Visual Studio możemy przystąpić do
instalacji narzędzi pomocniczych które pomogą nam w tworzeniu aplikacji.
Pierwszą niezbędną biblioteką jest .NET Compact Framework 2.0 SP1. Jest to
zestaw narzędzi wykorzystywanych do uruchamiania aplikacji na platformach
opartych o Windows Mobile. Aby ułatwić sobie proces budowania, debugowania i
uruchamiania aplikacji na urządzeniu konieczne jest zainstalowanie w systemie
Windows ActiveSync. Dzięki ActiveSync możliwe stanie się uruchamianie
projektowanej aplikacji, bezpośrednio z IDE, nie tylko na prawdziwym urządzeniu
ale również emulatorze. \\
\\
Ostatnim, ale i zarazem najważniejszym krokiem jest instalacja Windows Mobile
SDK. Na stronach firmy Microsoft dostępne są dwie wersje SDK, Standard oraz
Professional. Wersja Standard zawiera w sobie tylko wsparcie dla urządzeń z
Windows Mobile Classic lub Standard natomiast wersja Professional obejmuje
wszystkie dostępne środowiska. Wybór SDK można sprowadzić do następującej
zasady. Jeżeli zamierzamy tworzyć oprogramowanie dla urządzeń Smartphone bez
ekranu dotykowego w zupełności wystarczy nam wersja standardowa. Jeżeli
natomiast planujemy napisane aplikacje uruchamiać na PocketPC lub dotykowych
smartphonach będziemy potrzebować bibliotek systemu Windows Mobile Classic lub
Professional, tak więc konieczne jest użycie SDK w wersji Professional. Tak jak
w przypadku Visual Studio, również tutaj zaleca się instalację wszystkich
dostępnych na stronie producenta aktualizacji i poprawek. Jest to szczególnie
istotne podczas pracy z emulatorami urządzeń. Po zrealizowaniu tych kroków
otrzymujemy w pełni funkcjonalne środowisko do rozwoju aplikacji mobilnych dla
urządzeń smartphone.
\subsection{Środowisko uruchomieniowe}
