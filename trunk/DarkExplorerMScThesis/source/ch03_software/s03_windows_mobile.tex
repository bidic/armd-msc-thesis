\section{Platforma mobilna (Windows Mobile 6.1)}

Mobilne urządzenia przenośne z dnia na dzień zyskują na popularności. Każdego
dnia spotykamy się z nimi w domu, w pracy czy spacerując po parku. Z całą pewnością
można stwierdzić, iż większa część społeczeństwa obecnie jest w posiadaniu
telefonu komórkowego, komputera przenośnego czy też jakiegoś innego urządzenia
mobilnego. Wszystkie z wspomnianych urządzeń posiadają dedykowaną platformę
software'ową. Do najlepiej znanych we współczesnym świecie zaliczyć można między
innymi: Windows Mobile, iPhone, BlackBerry, Symbian OS, Android, Maemo, OpenMoko
itp. Każda z wymienionych platform posiada inną genezę jak również swoje mocne i
słabe strony.

Platformy takie jak Windows Mobile, BlackBerry czy iPhone ograniczone są do
urządzeń dedykowanych docelowo do współpracy z wspomnianymi środowiskami. Obok
różnorakich problemów z jakimi zmagają się wspomniane wcześniej platformy do
jednego z najpoważniejszych zaliczyć można bardzo ograniczone w niektórych
aspektach API. Nawet tak przenośna platforma jak Java na urządzeniach przenośnych
nie zawsze się sprawdza ze względu na liczne braki oraz różnice w API zmuszające
programistów do tworzenia kodu dedykowanego dla konkretnego urządzenia. Symbian
oraz Windows Mobile wypadają na tym tle nieco lepiej ponieważ wspierają szerszą
gamę urządzeń jak również ich API daje więcej możliwości niż ma to miejsce na
przykład w przypadku Javy. Głównym powodem takiego stanu rzeczy jest bardzo
szeroki i różnorodny asortyment platform sprzętowych utrudniający stworzenie
jednolitej i w pełni wykorzystującej wszystkie możliwości urządzenia platformy
programistycznej. Dostępne w chwili obecnej OpenSource'owe i wieloplatformowe
rozwiązania znajdują się ciągle we wczesnej fazie rozwoju i nie są jeszcze
powszechnie znane przez środowiska twórców oprogramowania.

Firma Microsoft wypuściła po raz pierwszy na światło dzienne swoją platformę
mobilną w latach 90-tych\cite{blog:wm-app-dev}. Natomiast w roku 2002 pojawiła
się pierwsza platforma Windows CE.NET. Zapoczątkowało to popularyzację urządzeń Pocket PC opartych o
system Windows CE 3.0 oraz późniejsze wersje. Dalszy rozwój bezprzewodowych
technologi telekomunikacyjnych pozwolił na integracje telefonu z komputerem
osobistym. Wspomniane urządzenia Pocket PC z 2002 roku wspierały między innymi
standard GSM, GPRS, bluetooth oraz umożliwiały użytkownikom dostęp do sieci
bezprzewodowych. W między czasie rozwojowi ulegały urządzenia typu SmartPhone
które koncepcyjnie były bardzo zbliżone do Pocket PC jednakże były one bardziej
zbliżone do telefonu niż komputera osobistego. Podstawową różnicą pomiędzy
Smartphone i Pocket PC jest fakt iż urządzenia Pocket PC posiadają ekran
dotykowy, a Smartphone wyposażone są jedynie w przyciski umożliwiające sterowanie
urządzeniem. Każde z tych urządzeń posiadało inny zestaw aplikacji pomocniczych
oraz wspierało inne standardy i technologie.
\newpage
W chwili obecnej większość urządzeń Pocket PC oraz Smartphone działają w oparciu
o system Windows Mobile 5 oraz Windows Mobile 6. Nowoczesne urządzenia Pocket PC
wyposażone są w procesor o taktowaniu 500-600 MHz oraz od 64-128 MB pamięci RAM.
Najnowsze urządzenia z tej grupy wyposażane są w 1 GHz procesor oraz 512 MB
pamięci.
 

\subsection{Środowisko rozwojowe}
Tworzenie aplikacji działających na urządzeniach pod kontrolą systemu Windows
Mobile jest niemal tak samo proste jak tworzenie zwykłych aplikacji na komputery
stacjonarne. Niemniej jednak do stworzenia w pełni funkcjonalnego środowiska
rozwojowego konieczne jest przejście przez klika kroków przygotowawczych
związanych z instalacją potrzebnych aplikacji narzędziowych. \\
\\
Przed rozpoczęciem przygody z tworzeniem aplikacji dla systemu Windows Mobile
konieczne jest zainstalowanie Microsoft Visual Studio. Zaleca się aby Visual
Studio było w wersji 2005 lub 2008. Niestety narzędzia umożliwiające rozwijanie
aplikacji mobilnych nie są poprawnie wykrywane przez Visual Studio 2010 oraz
poprzednie wydania w wersji Express. Dlatego też koniecznością jest instalacja
środowiska w wersji Standard lub Professional. Każda z tych wersji może zostać
pobrana w wersji czasowej ze stron firmy Microsoft lub w wersji pełnej z MSDNAA.
Visual Studio posłuży nam nie tylko do edycji kodu aplikacji ale pozwoli również
w prosty sposób budować, debugować oraz przygotować instalator finalnej wersji
aplikacji. Po poprawnym zainstalowaniu środowiska rozwojowego konieczne jest
pobranie i zainstalowanie dostępnych paczek serwisowych dostępnych dla wybranej
wersji Visual Studio. Pozwoli to uniknąć nieprzyjemnych niespodzianek podczas
instalacji bibliotek narzędziowych i późniejszej pracy.\\
\\
Jeżeli posiadamy już zainstalowaną kopię Visual Studio możemy przystąpić do
instalacji narzędzi pomocniczych które pomogą nam w tworzeniu aplikacji.
Pierwszą niezbędną biblioteką jest .NET Compact Framework 2.0 SP1. Jest to
zestaw narzędzi wykorzystywanych do uruchamiania aplikacji na platformach
opartych o Windows Mobile. Aby ułatwić sobie proces budowania, debugowania i
uruchamiania aplikacji na urządzeniu konieczne jest zainstalowanie w systemie
Windows ActiveSync. Dzięki ActiveSync możliwe stanie się uruchamianie
projektowanej aplikacji, bezpośrednio z IDE, nie tylko na prawdziwym urządzeniu
ale również emulatorze. \\
\\
Ostatnim, ale i zarazem najważniejszym krokiem jest instalacja Windows Mobile
SDK. Na stronach firmy Microsoft dostępne są dwie wersje SDK, Standard oraz
Professional. Wersja Standard zawiera w sobie tylko wsparcie dla urządzeń z
Windows Mobile Classic lub Standard natomiast wersja Professional obejmuje
wszystkie dostępne środowiska. Wybór SDK można sprowadzić do następującej
zasady. Jeżeli zamierzamy tworzyć oprogramowanie dla urządzeń Smartphone bez
ekranu dotykowego w zupełności wystarczy nam wersja standardowa. Jeżeli
natomiast planujemy napisane aplikacje uruchamiać na PocketPC lub dotykowych
SmartPhon'ach będziemy potrzebować bibliotek systemu Windows Mobile Classic lub
Professional, tak więc konieczne jest użycie SDK w wersji Professional. Tak jak
w przypadku Visual Studio, również tutaj zaleca się instalację wszystkich
dostępnych na stronie producenta aktualizacji i poprawek. Jest to szczególnie
istotne podczas pracy z emulatorami urządzeń. Po zrealizowaniu tych kroków
otrzymujemy w pełni funkcjonalne środowisko do rozwoju aplikacji mobilnych dla
urządzeń smartphone.

\subsubsection{Modele aplikacji}
Istnieje klika modeli rozwoju aplikacji dla Windows Mobile, a wybór docelowego
modelu został pozostawiony programiście. Pierwszy z nich służy do tworzenia
aplikacji w kodzie natywnym. Aplikacje pisane zgodnie z tym modelem cechują się
wysoką wydajnością, bezpośrednim dostępem do sprzętu oraz małym zużyciem
zasobów. Do rozwoju tego rodzaju aplikacji korzysta się z reguły ze środowiska do
rozwijania aplikacji z użyciem Embeded Visual C++. Główną wadą tego modelu jest
niska przenośność pomiędzy różnymi platformami zwłaszcza jeżeli aplikacja
korzysta z urządzeń specyficznych dla danego modelu urządzenia. Docelowo więc za
pomocą tego modelu tworzy się biblioteki i narzędzia ułatwiających tworzenie
bardziej skomplikowanych aplikacji. Jeżeli więc interesuje nas tworzenie
wysokopoziomowych aplikacji z GUI, skierowanych bezpośrednio do użytkowników,
zaleca się tworzenie tego typu aplikacji za pomocą kodu zarządzalnego z użyciem
takich języków jak C\# czy Visual Basic. Rozwijanie aplikacji w oparciu o kod
zarządzalny pozwala na stworzenie programu który będzie mógł w pełni
wykorzystywać możliwości oferowane przez Microsoft .NET Compact Framework.
Umożliwia to programiście tworzenie rozproszonych systemów mobilnych pracujących
zarówno w modelu ze stałym połączeniem jak i bez. Spora część narzędzi
dostępnych w ramach .NET Compact Framework jest również wykorzystywana do rozwoju
aplikacji na komputery stacjonarne. Biblioteka została zaprojektowana docelowo na
urządzenia o ograniczonych zasobach co w połączeniu z możliwościami języków z
rodziny .NET oraz integracją z Visual Studio daje nam profesjonalny zestaw
narzędzi do tworzenia aplikacji mobilnych.
\\
Trzecim modelem tworzenia programów pod Windows Mobile jest wykorzystywanie kodu
serwera do pracy z wieloma różnymi typami urządzeń poprzez jeden wspólny kod
bazowy i model cienkiego klienta. Oczywiście tego typu podejście ma sens jedynie
gdy możemy zagwarantować stabilny kanał komunikacyjny pomiędzy urządzeniem
klienta, a serwerem. Każdy z przedstawionych modeli idealnie sprawdza się jeżeli
tylko w prawidłowy sposób wybierzemy model najbardziej odpowiadający potrzebom
naszej aplikacji.

\subsubsection{Graficzny interfejs użytkownika}
Dzięki wygodnemu systemowi projektowania GUI
dostępnego w Visual Studio tworzenie interfejsu
użytkownika dla platformy mobilnej jest niemal tak proste jak w przypadku
tradycyjnych aplikacji, a jedyną różnicą jest ilość i rodzaj dostępnych
kontrolek. Różnice te wynikają z faktu iż niektóre urządzenia mobilne posiadają
ekrany dotykowe, a inne nie. Co za tym idzie, rozwój interfejsu użytkownika
staje się bardziej skomplikowany zwłaszcza jeżeli interesuje nas rozwój
aplikacji wspólnej dla obydwóch platform sprzętowych. W tym miejscu nie może
zabraknąć informacji, że oprogramowanie zbudowane dla PocketPC nie uruchomi się
na urządzeniach SmartPhone natomiast sytuacja odwrotna jest możliwa do momentu w
którym aplikacja nie zacznie korzystać ze specyficznych funkcji SmartPhone.
Naturalnym stanem rzeczy wydaje się fakt, iż wiele funkcji i komponentów
graficznych znanych z aplikacji desktopowych zostało usunięte z bibliotek
Windows Mobile, aby zapewnić jej wydajność i niewielki rozmiar. Dlatego też
pozostawiono tylko niezbędne, najprostsze komponenty. Ponieważ wydajność  i
zasoby pamięciowe urządzeń stale rosną, również ilość narzędzi dostępnych w SDK
jest systematycznie zwiększana, a co za tym idzie różnice pomiędzy kolejnymi
wersjami .NET Compact Framework są bardzo duże. Dlatego też posiadanie jak
najbardziej aktualnej wersji SDK ma niebagatelne znaczenie dla wygody tworzenia
aplikacji. Podsumowując rozwój GUI dla platformy mobilnej nie różni się bardzo
od tworzenia interfejsu użytkownika dla aplikacji dekstopowej. Istnieje również
możliwość rozwijania GUI w oparciu o silniki 3D. W chwili obecnych dostępne są
takie rozwiązania jak GAPI (Game API), OpenGL ES (Embeded Systems), Open VG
(Vector Grapics). Jednakże rozwój takich aplikacji jest niezwykle trudny gdyż
wymaga od programisty tworzenie maksymalnie optymalnego kodu, ze względu na
ograniczone możliwości niektórych urządzeń. 

\subsubsection{Komunikacja}
Nowoczesne urządzenia mobilne posiadają szeroką gamę możliwości komunikacyjnych.
Posiadają one dostęp do szybkich sieci bezprzewodowych w standardzie 802.11
WiFi. Umożliwiają one również komunikację za pomocą portu podczerwieni,
bluetooth czy USB. Podejmując decyzję na temat wyboru kanału komunikacji należy
brać pod uwagę nie tylko parametry techniczne, ale również liczbę standardów i
protokołów dostępnych dla danego kanału komunikacji. Firma Microsoft dostarcza
szereg interfejsów programistycznych (API) umożliwiających niemal błyskawiczny
dostęp do funkcji komunikacyjnych urządzenia. AcitveSync API dostarcza funkcjonalność
umożliwiającą komunikację za pomocą protokołu synchronizacji. Natomiast
Bluetooth API dostarcza zestaw narzędzi umożliwiających nawiązywanie komunikacji
bezprzewodowej pomiędzy telefonami jak i urządzeniami peryferyjnymi.  API
Managera połączeń dostarcza zestaw usług umożliwiających automatyzację procesu
nawiązywania połączenia oraz zarządzanie ich aktywnością. API wymiany obiektów
(OBEX API) dostarcza funkcjonalność umożliwiającą wymianę danych pomiędzy
urządzeniami za pomocą efektywnego, kompaktowego protokołu binarnego
dedykowanego dla urządzeń z ograniczonymi zasobami. Remote API (RAPI) dostarcza
funkcję do zarządzania oraz zdalnego wywoływania metod po stronie
urządzenia klienta. Dostępne są m.in. funkcje dostępu do rejestru, plików, bazy
danych czy też konfiguracji urządzenia. Najważniejszą funkcjonalnością jest
jednak możliwość zdalnego wywoływania procedur. Za pomocą funkcji
CeRapiInvoke() przesyłamy do urządzenia nazwę biblioteki dynamicznej wraz z
nazwą metody która ma zostać wywołana na urządzeniu mobilnym. Kolejnym zestawem
narzędzi jest Pocket Outlook Object Model API dostarcza funkcje do zarządzania
obiektami Pocket Outlook  co umożliwia synchronizację zadań, kalendarza czy
kontaktów za pomocą prostego i intuicyjnego interfejsu. Dostępne jest również
Telephony API (TAPI) które zawiera w sobie biblioteki umożliwiające zarządzanie
kartą SIM oraz wiadomościami SMS. TAPI udostępnia również zestaw funkcji
umożliwiających dostęp do funkcji telefonowania oraz protokołu WAP. Nie
zabrakło również narzędzi do pracy z portami USB oraz COM. Część z dostępnych portów COM
jest zarezerwowana dla urządzeń wewnętrznych, ale pozostałe dostępne są do
pełnej dyspozycji użytkownika.

\subsubsection{Debugowanie}
Microsoft Visual Studio umożliwia debugowanie aplikacji działających pod
kontrolą Windows Mobile w taki sam sposób jak ma to miejsce w przypadku
tradycyjnych aplikacji desktopowych. Ponadto programista ma do swojej
następujące narzędzia: emulator, panel zarządzania emulowanymi urządzeniami,
panel punktów przerwań i wątków. Niestety w Visual Studio nie uda się nam
jednocześnie debugować kodu natywnego i zarządzalnego. Możliwe jest natomiast
uruchomienie zarówno projektu napisanego w Visual C++ jak i projektu opartego o
kod zarządzalny, a dzięki funkcjonalności ,,Dołącz do procesu'' możliwe jest
zdalne dołączenie się doi monitorowanie procesu działającego na urządzeniu lub
emulatorze urządzenia. Narzędziem umożliwiającym komunikację pomiędzy
urządzeniem a systemem jest ActiveSync instalowany wraz ze środowiskiem
rozwojowym.  Za pomocą narzędzia ActiveSync możemy łączyć się nie tylko z
rzeczywistymi urządzeniami ale również z emulatorami. Umożliwia to pełną
wirtualizacje urządzeń mobilnych i znacznie ułatwia testowanie funkcjonalności
zwłaszcza pomiędzy różnymi platformami urządzeń (SmartPhone, PocketPC). Jedynym
ograniczeniem tego procesu jest możliwość utrzymania tylko jednego aktywnego
połączenia co uniemożliwia debugowanie na wielu urządzeniach jednocześnie. Co
więcej Visual Studio umożliwia debugowanie jedynie aplikacji stworzonej przez
programistę, nie możliwa jest z poziomu IDE debugowanie aplikacji i usług
systemowych działających na urządzeniu. Do tego typu debugowania konieczne
byłoby zbudowanie własnej wersji systemu Windows Mobile przy użyciu Platform
Buildera. Narzędzie to umożliwia również tworzenie własnego SDK dla Visual
Studio i platformy Windows CE. Dodatkową możliwością dostępną z poziomu
emulatora jest emulowanie połączenia z siecią GSM oraz wsparcie dla GPS.
Umożliwia to testowanie, debugowanie i rozwijanie szerokiego spektrum aplikacji
bez konieczności posiadania urządzenia fizycznie.

\subsection{Środowisko uruchomieniowe}
