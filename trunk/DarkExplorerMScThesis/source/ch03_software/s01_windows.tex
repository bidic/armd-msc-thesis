\section{Platforma PC dla systemu Windows}
\subsection{Środowisko tworzenia oprogramowania}
\subsection{Środowisko tworzenia programu sterującego}
Pierwszym krokiem do przygotowania środowiska rozwojowego umożliwiającego
rozwijanie oprogramowania sterującego robotem jest instalacja sterowników
wymaganych przez system Windows do obsługi interfejsu JTAG za pomocą którego
odbywa się proces wgrywania przygotowanego oprogramowania do pamięci robota.
Kolejnym wymaganym krokiem jest instalacja i konfiguracja narzędzi
umożliwiających stworzenie pliku binarnego umożliwiającego uruchomienie
przygotowanej na platformie sprzętowej robota. Ostatnim etapem przygotowań jest
instalacja oprogramowania umożliwiającego programowanie układu za pomocą
wspomnianego interfejsu JTAG oraz debugowanie aplikacji w trakcie jej działania.
\subsubsection{Instalacja WinARM}
Do kompilacji kodu źródłowego oprogramowania zajmującego się sterowaniem
podzespołami robota wykorzystany został zestaw narzędzi znany pod nazwą WinARM.
WinARM jest zestawem narzędzi umożliwiających tworzenie oprogramowania dla
kontrolerów opartych na platformie ARM. W odróżnieniu od innych dostępnych
obecnie narzędzi środowisko WinARM nie wymaga dodatkowej instalacji środowiska
MinGW czy też Cygwina. Wszystkie potrzebne narzędzia dostarczane są w ramach
środowiska WinARM. Narzędzia WinARM pomyślnie przeszła testy z kontrolerami Atmel
AT91SAM7S64, AT91SAM7S256, AT91RM9200 ARM7TDMI oraz Philips LPC2106, Philips
LPC2129, Philips LPC2138, Philips LPC2148. Narzędzia dostarczane w ramach
środowiska WinARM kompilatory i narzędzia powinny prawidłowo współpracować ze
wszystkimi mikrokontrolerami opartymi o architekturę ARM(-TDMI/Thumb
itp.).\newline \newline Instalację środowiska WinARM należy rozpocząć od pobrania
archiwum z najnowszą wersją narzędzi ze strony
\url{http://gandalf.arubi.uni-kl.de/avr_projects/arm_projects/}. W chwili pisania
pracy dostępna była wersja środowiska WinARM w wersji 20060606. Po zakończeniu
procesu pobierania, archiwum należy rozpakować w taki sposób aby wszystkie
podstawowe narzędzia dostępne były w katalogu \url{C:\WinARM\bin}. Umieszczenie
katalogu z narzędziami WinARM w innej lokalizacji jest również możliwe, ale może
wymagać wykonania dodatkowych operacji konfiguracyjnych w celu zapewnienia
poprawności działania wszystkich narzędzi. Aby udostępnić narzędzia WinARM z
linii poleceń systemu Windows konieczne jest dodanie do zmiennej systemowej PATH
następujących wartości \url{C:\WinARM\bin;C:\WinARM\utils\bin;}.
\subsubsection{Instalacja sterowników programatora}
Programowanie oraz debugowanie aplikacji robota może zostać zrealizowane za
pomocą dowolnego programatora kompatybilnego z interfejsem JTAG. Programatory
oparte o interfejs LPT nie wymagają dodatkowej konfiguracji w systemie Windows.
Nieco inaczej wygląda sytuacja z programatorami opartymi o interfejs USB które
wymagają przed pierwszym użyciem zainstalowania sterowników umożliwiających
prawidłowe rozpoznanie programatora przez system. Jednym z bardziej popularnych
programatorów USB jest TriTon JTAG. TriTon JTAG to interfejs JTAG przeznaczony
dla procesorów zbudowanych w oparciu o rdzeń ARM podłączany do komputera za
pomocą portu USB. TriTon JTAG posiada standardowe 20 pinowe złącze JTAG wraz z
wyprowadzeniami sygnałów RxD i TxD interfejsu UART. TriTon JTAG współpracuje z
OpenOCD, pozwalając na programowanie oraz debugowanie działającej aplikacji.
Urządzenie oparte jest o układ FT2232 który umożliwa jego współprace także z
innymi środowiskami rozwoju oprogramowania dla platformy ARM wspierającymi
wspomniany układ. \newline \newline Instalację sterowników do programatora należy
rozpocząć od pobrania sterowników do układu FT2232 ze strony
\url{http://www.ethernut.de/en/download/}. Na stronie dostępne są sterowniki
przeznaczone dla systemów Windows 2000, XP, Server 2003, Vista oraz Server 2008.
Po rozpakowaniu archiwum ze sterownikami należy za pomocą interfejsu USB
podłączyć programator do komputera. Po wykryciu system Windows trzykrotnie
poprosi o podanie ścieżki do sterowników do urządzeń Triton JTAG, Triton USB
RS232 Adapter oraz USB Serial Port. Należy wtedy skazać ścieżkę do katalogu w
którym rozpakowane zostały sterowniki pobrane ze strony wspomnianej wcześniej.
Po poprawnym zakończeniu instalacji w Menadżerze urządzeń systemu Windows
windoczne będą następujące elementy
\begin{itemize}
  \item Triton USB JTAG Adapter,
  \item Triton USB RS232 Adapter,
  \item Triton JTAG
\end{itemize}
\subsubsection{Instalacja Open On-Chip Debugger}
OpenOCD zostało zapoczątkowane przez Dominika Rath w ramach pracy dyplomowej
relizowanej na uniwersytecie w Augsburg. Od tamtego czasu OpenOCD bardzo się
rozwinęło i urosło do rozmiarów aktywnego projektu open-sourcowego wspieranego
przez programistów z całego świata. Celem OpenOCD jest dostarczenie
uniwersalnego narzędzia umożliwiającego debugowanie i programowanie systemów
wbudowanych. \newline \newline
Strona projektu OpenOCD dostępna jest pod adresem
\url{http://openocd.berlios.de/web/}. Dostępne są tam zarówno źródła jak i
dokumentacja do projektu. Niestety w chwili pisania pracy magisterskiej autorzy
nie udostępniali wersji skompilowanej dla systemu Windows. Dlatego też użyta
została niezależna wersja OpenOCD z przygotowanym instalatorem dla systemu
Windows. Instalator OpenOCD dla Windows jest do pobrania ze strony
\url{http://www.ethernut.de/en/download/}. Po uruchomieniu instalatora wybieramy
lokalizację docelową w której OpenOCD ma zostać zainstalowane. Po zakończeniu
instalacji konieczne jest uzupełnienie wartości zmiennej systemowej PATH ścieżką
do miejsca instalacji OpenOCD, domyślnie \url{C:\ethernut\nut\tools\win32}.

