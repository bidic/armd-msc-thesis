\section{Biblioteka programistyczna dla platformy .NET}
Jednym z poważniejszych problemów zauważonych podczas analizy pierwotnej
konfiguracji robota był brak bibliotek umożliwiających tworzenie oprogramowania
które pozwalałoby na dowolne wykorzystanie możliwości oferowanych przez
konfigurację sprzętową robota. Brak tego rodzaju narzędzi znacząco ogranicza
możliwości rozwoju robota gdyż każda próba tworzenia nowego oprogramowania wymaga
od programisty zagłębiania się w szczegóły implementacji systemu wbudowanego
który kontroluje działanie robota. Aby usunąć tak poważne ograniczenie
zaprojektowana została biblioteka mająca na celu udostępnienie narzędzi
pozwalających programiście na skupienie się jedynie wysokopoziomowej
funkcjonalności bez konieczności szczegółowej analizy protokołu komunikacji i
zasad działania poszczególnych funkcji robota.

Przed przystąpieniem do implementacji konieczne jest podjęcie rozważnej decyzji z
wyborem środowiska i języka programowania w jakim biblioteka zostanie napisana.
Biorąc pod uwagę ciągle rosnącą popularność obiektowych języków programowania
oczywistym wydaje się być wybór języka właśnie z tej rodziny. Decydującym
aspektem, wpływającym na ostateczny wybór docelowej platformy rozwojowej, jest
więc ilość dostępnych bibliotek oraz przenośność kodu pomiędzy dostępnymi na
rynku platformami sprzętowymi. Po przeprowadzeniu wnikliwej analizy ostateczny
wybór padł na język C\# oraz platformę .NET. Wybór motywowany jest faktem iż
platforma .NET jest jednym z najdynamiczniej rozwijających środowisk
programistycznych ostatnich lat. Firma Microsoft dostarcza szereg bibliotek
dodatkowych oraz narzędzi pozwalających na szybkie tworzenie oprogramowania
działającego zarówno na urządzeniach mobilnych jaki i stacjonarnych. Dzięki
pracy programistów w ramach projektu Mono\footnote{Więcej informacji na temat projektu
Mono można znaleźć pod adresem strony internetowej http://www.mono-project.com/}
powstała platforma umożliwiająca uruchamianie aplikacji napisanych w języku C\#
nie tylko pod kontrolą systemu Windows, ale również pod systemami z rodziny Linux
i Mac. Dodatkowym atutem platformy .NET jest bardzo duża liczba bibliotek
dodatkowych dostarczanych przez środowiska programistów opensource.

W ramach pracy magisterskiej zaprojektowana została biblioteka programistyczna w
języku C\#. Docelową platformą uruchomieniową dla przygotowanej biblioteki są
systemy z rodziny Windows, Windows Mobile oraz Linux. Przy tworzeniu biblioteki
brane pod uwagę były najnowsze trendy w dziedzinie programistycznych wzorców
projektowych przy jednoczesnym zachowaniu spójności i uniwersalności kodu dla
poszczególnych środowisk uruchomieniowych. W wyniku implementacji powstała
wielowątkowa biblioteka oparta na zdarzeniach pozwalająca na dostęp do wszystkich
funkcji robota za pomocą intuicyjnego interfejsu programistycznego. Biblioteka
udostępnia szerokie spektrum metod pozwalających na swobodne sterowanie i
zarządzanie dostępnymi funkcjami robota. Wśród metod bibliotecznych znaleźć można
funkcje pozwalające na bezpośrednią interakcję z poszczególnymi podzespołami
bazowymi, jak również takie które umożliwiają wykonywanie predefiniowanych
sekwencji zadań przewidzianych przez autorów projektu. Biblioteka pokrywa swoją
funkcjonalnością nie tylko wsparcie dla wszystkich dostępnych roszerzeń
sprzętowych robota, ale również dostarcza interfejs pozwalający na automatyczną
obsługę połączenia z robotem z wykorzystaniem technologii bluetooth. Fakt ten
jest o tyle istotny, że proces komunikacji okazał się być najbardziej wrażliwym
elementem podczas migracji biblioteki pomiędzy poszczególnymi platformami
softwareowymi. Szczegółowa lista wszystkich dostępnych funkcji wraz z niezbędnym
komentarzem zamieszczona została w dodatku poświęconym kodu źródłowemu
stworzonego w ramach pracy magisterskiej. Taki sposób implementacji pozwala na
tworzenie oprogramowania współpracującego z nową wersją robota nawet przez osoby
nie posiadające dostatecznej wiedzy i umiejętności tworzenia oprogramowania do
obsługi systemów wbudowanych.

W ramach pracy magisterskiej stworzona została również przykładowa aplikacja
sterująca prezentująca wszystkie możliwości oferowane zarówno przez warstwę
aplikacyjną jak i sprzętową robota Dark Explorer. 

\textcolor{red}{TODO: zrzut ekranu aplikacji sterującej dla .net'a}