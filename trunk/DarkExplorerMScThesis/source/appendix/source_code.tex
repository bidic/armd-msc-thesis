\label{ch:source-code}
W tym załączniku zamieszczony został kod źródłowy części oprogramowania,
które zostało stworzone w trakcie realizacji pracy magisterskiej. Wszystkie
zamieszczone w tym rozdziale fragmenty kodu, mają charakter poglądowy i służyć
mogą jedynie jako podstawowa dokumentacja, pozwalająca zrozumieć architekturę
zaimplementowanych rozwiązań. Rozdział ten należy traktować jako
uzupełnienie informacji przedstawionych w~ramach poprzednich rozdziałów.
Czytelnik zainteresowany dostępem do pełnego kodu źródłowego wraz z dokumentacją
programisty, może takowy znaleźć na dołączonym do pracy nośniku CD. Wspomnianą
dokumentację wraz z najaktualniejszą wersją źródeł można również pobrać ze
strony \url{http://lumifun.ftj.agh.edu.pl/doku.php?id=user:hanusiak_nowacki:start}.

\section{Dokumentacja kodu źródłowego robota}
\section{Dokumentacja biblioteki dla języka Java}
\lstinputlisting[language=Java,caption=Źródła klasy sterującej
(Controller)]{appendix/src/java/Controller.java} 
\lstinputlisting[language=Java,caption=Źródła klas obsługi połączenia bluetooth
(BTConnection)]{appendix/src/java/BTConnection.java}
\section{Dokumentacja biblioteki dla języka C\#}
\lstinputlisting[language=CSharp,caption=Źródła klasy sterującej
(DarkExplorer.cs)]{appendix/src/csharp/DarkExplorer.cs} 
\lstinputlisting[language=CSharp,caption=Źródła klas obsługi protokołu
komunikacji (DarkExplorerComm.cs)]{appendix/src/csharp/DarkExplorerComm.cs}
\lstinputlisting[language=CSharp,caption=Źródła klas typów pomocniczych
(DarkExplorerTypes.cs)]{appendix/src/csharp/DarkExplorerTypes.cs}
\lstinputlisting[language=CSharp,caption=Źródła klas zdarzeń
(DarkExplorerEvents.cs)]{appendix/src/csharp/DarkExplorerEvents.cs} 