\section{Aplikacja dla komputerów stacjonarnych}
\label{sec:java-app}
Na potrzeby prezentacji biblioteki zarządzającej robotem Dark Explorer napisanej
w języku Java, stworzono aplikację dla komputerów stacjonarnych. Można ją
uruchomić na wielu systemach operacyjnych, między innymi na systemach Windows
oraz Linux. Aplikacja demonstracyjna korzysta z przygotowanej na rzecz tej pracy
magisterskiej biblioteki zarządzającej robotem oraz z biblioteki wspierającej JSR
82. Z pośród wielu bibliotek przedstawionych w tabeli \ref{tab:JSR82SDK} wybrana
została bibloteka BlueCove\cite{website:bluecove.org}. Działa ona pod wieloma
systemami operacyjnymi, jest darmowa dla zastosowaniach niekomercyjnych, oraz
posiada dokumentacje wraz z przykładowymi aplikacjami.

Przygotowany program zarządzający robotem pozwala na zaprezentowanie wszystkich
jego funkcjonalności stworzonych podczas rozwoju tej pracy magisterskiej. Rysunek
\textcolor{red}{TODO: zrzut ekranu aplikacji sterującej dla Javy} przedstawia
ekran główny wspomnianej aplikacji.
