\section{Algorytm rekonstrukcji ścieżki powrotnej}
\label{sec:rtrwca}
Robot Dark Explorer został wyposażony w czujniki, które mają na celu dostarczenie
informacji niezbędnych do ustalenia toru ruchu robota. Niniejszy podrozdział
opisuje algorytm wykorzystujący dane z czujników w celu wyznaczenia trasy od
obecnego położenia do miejsca z którego robot został przyniesiony przez
operatora.

Algorytm rekonstrukcji ścieżki powrotnej składa się z dwóch części. Pierwsza z
nich odpowiedzialna jest za zapamiętywanie toru ruchu robota, druga natomiast za
odtworzenie ścieżki na podstawie zgromadzonych danych. Cały algorytm opiera się
na informacjach uzyskanych z dwóch czujników: żyroskopu oraz akcelerometru.
Czujnik przyspieszenia został zastosowany w celu określenia odległości jaką
przebył robot niesiony przez operatora. Natomiast żyroskop pozwala na uzyskanie
informacji o zmianie kierunku.

W pierwszym podejściu do rozwiązania problemu wykrycia toru ruchu ciała
przenoszonego przez operatora, odległość miała być obliczana bezpośrednio z
informacji o przyspieszeniu uzyskanej z akcelerometru. W celu obliczenia drogi po
jakiej poruszało się ciało, mając jedynie informacje o przyspieszeniu, konieczne
jest wykonanie dwukrotnego całkowania wartości otrzymanych z czujnika. Niestety
operacja ta wymaga bardzo krótkiego okresu czasu pomiędzy pomiarami w celu
zminimalizowania błędów całkowania. Istotna jest także wysoka czułość i
dokładność samego czujnika przyspieszenia. Po wykonaniu wstępnych testów tego
rozwiązania stwierdzono iż odległości zmierzone za pomocą tej metody są
wystarczająco dobre.

Finalnym rozwiązaniem okazała się metoda wykrywania ilości kroków które wykonał
operator robota podczas przemieszczania go w inne miejsce. Podczas wykonywania
kroków, osoba trzymająca robota wykonuje samoczynnie minimalne ruchy ręką w górę
i w dół. Dzięki wykrywaniu odpowiedniej sekwencji przyspieszeń jesteśmy w stanie
określić czy operator wykonał krok. Po wykryciu każdego kroku zapisywana jest
wartość kąta pomiędzy kierunkiem ruchu z poprzedniego i obecnego kroku. W taki
sposób cała trasa zapisywana jest jako lista zawierająca informacje o zmianie
kierunku po wykonaniu przemieszczenia o odległość jednego kroku.
