\section{Biblioteka AT91LIB}
\label{sec:at91lib}
Programowanie mikrokontrolerów z rodziny ARM w ogólnym zarysie sprowadza się do odpowiedniego zarządzania rejestrami.  W celu ułatwienia tego zadania programiście, stworzono bibliotekę która opakowuje proces przypisywania bitów do odpowiednich miejsc w rejestrach i udostępnia zrozumiałe dla człowieka funkcje, struktury i predefiniowane wartości. 

\begin{table}[!ht]
\rowcolors{2}{white}{gray!20}
\centering
\caption{Opis poszczególnych katalogów biblioteki AT91LIB v.1.5}
   	\begin{tabular}{ | c | c | p{1.75cm} |} \hline
   		Nazwa katalogu & Opis \\ \hline
   		boards & obsługa płyt ewaluacyjnych \\
		 & oraz modułów mikrokontrolerów \\ \hline
   		components & dodatkowe komponenty zewnętrzne \\
		 & takie jak np. kontroler ethernet \\ \hline
		drivers & wyspecjalizowane wysokopoziomowe \\ 
		 & funkcje zarządzające działaniem \\
		 & urządzeń peryferyjnych \\ \hline
		memories & obsługa różnego \\ 
		 & rodzaju pamięci \\ \hline
		peripherals & funkcje niskiego \\
		 & poziomu zarządzające \\ 
		 & urządzeniami peryferyjnymi \\ \hline
		usb & obsługa USB \\ \hline
		utility & narzędzia oraz \\ 
		 & algorytmy dodatkowe \\ \hline
   	\end{tabular}
\label{tab:AT91LIB}
\end{table}

Biblioteka o której mowa to AT91LIB\cite{AT91LIB} v.1.5 zaprojektowana przez firmę Atmel na potrzeby mikrokontrolerów ARM. Dzięki jej zastosowaniu nie jest konieczne dokładne zaznajomienie się ze strukturą rejestrów mikrokontrolera, przez co programista może skupić swoją uwagę na implementacji konkretnego rozwiązania. Udostępnia ona także mechanizm informujący programistę o wykonanym przez niego błędzie logicznym. Przykładem takiego błędu może być próba wykorzystania urządzenia peryferyjnego obsługującego interfejs TWI bez wcześniejszej inicjalizacji zegara wymaganego przez to urządzenie. W przypadku wykrycia takiej sytuacji AT91LIB, o ile to możliwe, poinformuje nas o błędzie, a następnie przerwie wykonywanie programu.

\begin{figure}[!ht]
 \centering
 \includegraphics[height=40mm]{../images/ch05/at91lib_dirs.png}
 \caption{Struktura katalogów biblioteki AT91LIB v.1.5}
 \label{fig:AT91LIB}
\end{figure}

Biblioteka AT91LIB jest podzielona na 7 katalogów (rys. \ref{fig:AT91LIB}) odpowiedzialnych za zarządzanie poszczególnymi elementami związanymi z mikrokontrolerem. W tabeli \ref{tab:AT91LIB} przedstawiono opis funkcjonalności obsługiwanych przez kod zawarty w poszczególnych katalogach biblioteki.

Zastosowanie AT91LIB pozwoliło na stworzenie przejrzystego kodu sterującego robotem, który będzie można modyfikować bez dokładnego zagłębiania się w notę katalogową mikrokontrolera AT91SAM7S256.