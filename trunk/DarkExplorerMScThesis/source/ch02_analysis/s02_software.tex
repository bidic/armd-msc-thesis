\section{Analiza oprogramowania}
\subsection{Firmware}
Jako firmware określane jest oprogramowanie którego komendy są wykonywane przez mikrokontroler. W tym podrozdziale zajmiemy się kodem źródłowym opisującym sposób działania robota.

Pliki składające się na kod źródłowy robota możemy podzielić na dwa rodzaje:
\begin{itemize}
 \item napisane przez autora
 \item dostarczone przez producenta mikrokontrolera
\end{itemize}

Pliki dostarczone przez producenta mikrokontrolera są po prostu biblioteką, dzięki której programista może kontrolować działanie różnych urządzeń peryferyjnych mikrokontrolera. Biblioteka wykorzystana przez twórcę Dark Explorera bazuje na prostych funkcjach wpisujących podawane wartości do odpowiednich rejestrów mikrokontrolera. Można powiedzieć, że jest to biblioteka nisko poziomowa. Wymaga ona dużej wiedzy na temat urządzeń peryferyjnych z których mamy zamiar korzystać oraz rejestrów które nimi kontrolują. Praktycznie nie jest możliwe oprogramowanie robota przy pomocy tej biblioteki bez wcześniejszego dokładnego zaznajomienia się z notą katalogową mikrokontrolera AT91Sam7S256. Podczas rozwoju robota przydatnym byłoby wykorzystanie innej biblioteki, która jest bardziej przyjazna dla programisty.

Bazowa wersja firmware'u napisanego przez autora Dark Explorera została podzielona na sześć plików:
\begin{itemize}
 \item board.h -- plik nagłówkowy z informacjami dotyczącymi mikromodułu mikrokontrolera
 \item pio.h -- definicje określające wejścia i wyjścia ogólnego przeznaczenia zamontowane na płycie głównej robota
 \item main.c -- inicjalizacje początkowe, obsługa przerwań systemowych, interpretacja komend bluetooth, pętla główna programu
 \item peripherals.c -- funkcje do obsługi urządzeń peryferyjnych
 \item rozpoznawanie.c -- analiza i rozpoznawanie obrazu
 \item utils.c -- procedury sterujące i obliczeniowe wyższego poziomu
\end{itemize}

Biorąc pod uwagę ilość kodu znajdującą się w plikach, taki podział wydaje się być wystarczający. W przyszłości trzeba zadbać o gęstsze partycjonowanie kodu źródłowego na logiczne bloki w celu zapewnienia przejrzystości kodu źródłowego.

\subsection{Aplikacja zarządzająca}
Do kontrolowania robota została stworzona aplikacja graficzna działająca pod systemem Windows. Interfejs graficzny aplikacji jest atrakcyjny i przejrzysty. Pozwala ona na kontrole następujących funkcji robota:
\begin{itemize}
 \item kontrola kierunku i szybkości jazdy robota za pomocą wektora wodzącego oraz klawiatury
 \item kontrola wierzy obserwacyjnej
 \item włączanie oraz wyłączanie diody oświetlającej
 \item informacja o stanie akumulatorów
 \item zarządzanie trybem autonomicznym robota
 \item konfiguracja oraz status połączenia bluetooth\\
\end{itemize}

\begin{figure}[!ht]
 \centering
 \includegraphics[height=85mm]{../images/ch02/decontrollprogram.png}
 \caption{Okno aplikacji zarządzającej Dark Explorera}
 \label{fig:AplikacjaZarz}
\end{figure}

Aplikacja zarządzająca została napisana tylko na jeden rodzaj systemu operacyjnego. Nie ma możliwości kontrolowania robota przy pomocy komputera z zainstalowanym systemem operacyjnym innym niż Microsoft Windows. Dlatego też, konieczne będzie stworzenie aplikacji zarządzającej, którą będzie można modyfikować oraz uruchomić na dowolnym systemie.

W celu komunikacji z robotem, aplikacja zarządzająca wysyła do robota odpowiednie komunikaty. Komunikaty te składają się z pojedynczego znaku i liczby. Możliwe, że w przypadku obsługi większej ilości urządzeń na Dark Explorerze, będzie konieczne stworzenie bardziej zaawansowanego protokołu komunikacyjnego. Zapewni to możliwość sprawdzenia czy dana komenda jest poprawnie skonstruowana, czy może zawiera jakieś błędy które pojawiły się podczas transmisji.

Zarówno w aplikacji zarządzającej jak i na robocie został zastosowany mechanizm retransmisji pakietów z obrazem w przypadku błędów w trakcie połączenia. Jest to bardzo dobry pomysł w szczególności w przypadku przesyłania takiej ilości danych jaką generuje kamera. 