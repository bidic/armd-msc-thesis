\section{Platforma mobilna (Windows Mobile 6.1)}

Mobilne urządzenia prznośne z dnia na dzień zyskują na popularności. Każdego dnia
spotykamy się z nimi w domu, w pracy czy spacerując po parku. Z całą pewnością
można stwierdzić, iż większa część społeczeństwa obecnie jest w posiadaniu
telefonu komórkowego, komputera przenośnego czy też jakiegoś innego urządzenia
mobilnego. Wszystkie z wspomnianych urządzeń posiadają dedykowaną platformę
software'ową. Do najlepiej znanych we współczesnym świecie zaliczyć można między
innymi: Windows Mobile, iPhone, BlackBerry, Symbian OS, Android, Maemo, OpenMoko
itp. Każda z wymienionych platform posiada inną genezę jak również swoje mocne i
słabe strony.

Platformy takie jak Windows Mobile, BlackBerry czy iPhone ograniczone są do
urządzeń dedykowanych docelowo do współpracy z wspomnianymi środowiskami. Obok
różnorakich problemów z jakimi zmagają się wspomniane wcześniej platformy do
jednego z najpoważniejszych zaliczyć można bardzo ograniczone w niektórych
aspektach API. Nawet tak przenośna platforma jak Java na urządzeniach przenośnych
nie zawsze się sprawdza ze względu na liczne braki oraz różnice w API zmuszające
programistów do tworzenia kodu dedykowanego dla konkretnego urządzenia. Symbian
oraz Windows Mobile wypadają na tym tle nieco lepiej ponieważ wspierają szerszą
gamę urządzeń jak również ich API daje więcej możliwości niż ma to miejsce na
przykład w przypadku Javy. Głównym powodem takiego stanu rzeczy jest bardzo
szeroki i różnorodny asortyment platform sprzętowych utrudniający stworzenie
jednolitej i w pełni wykorzystującej wszystkie możliwości urządzenia platformy
programistycznej. Dostępne w chwili obecnej OpenSource'owe i wieloplatformowe
rozwiązania znajdują się ciągle we wczesnej fazie rozwoju i nie są jeszcze
powszechnie znane przez środowiska twórców oprogramowania.

Firma Microsoft wypuściła po raz pierwszy na światło dzienne swoją platformę
mobilną w latach 90-tych. Natomiast w roku 2002 pojawiła się pierwsza platforma
Windows CE.NET. Zapoczątkowało to popularyzację urządzeń Pocket PC opartych o
system Windows CE 3.0 oraz późniejsze wersje. Dalszy rozwój bezprzewodkowych
technologi telekomunikacyjnych pozwolił na integracje telefonu z komputerem
osobistym. Wspomniane urządzenia Pocket PC z 2002 roku wspierały między innymi
standard GSM, GPRS, bluetooth oraz umożliwiały użytkownikom dostęp do sieci
bezprzewodowych. W między czasie rozwojowi ulegały urządzenia typu SmartPhone
które koncepcyjnie były bardzo zbliżone do Pocket PC jednkaże były one bardziej
zbliżone do telefonu niż komputera osobistego. Podstawową różnicą pomiędzy
Smartphone i Pocket PC jest fakt iż urządzenia Pocket PC posiadają ekran
dotykowy, a Smartphone wyposażone są jedynie w przyciski umożliwiające sterowanie
urządzeniem. Każde z tych urządzeń posiadało inny zestaw aplikacji pomocniczych
oraz wspierało inne standardy i technologie.

W chwili obecnej większość urządzeń Pocket PC oraz Smartphone działają w oparciu
o system Windows Mobile 5 oraz Windows Mobile 6. Nowoczesne urządzenia Pocket PC
wyposażone są w procesor o taktowaniue 500-600 MHz oraz od 64-128 MB pamięci RAM.
Najnowsze urządzenia z tej grupy wyposażane są w 1 GHz procesor oraz 512 MB
pamięci.

Development Tools

There are a few models of development of applications for Windows Mobile:

Win32 API MFC .NET Compact Framework.

Microsoft Co gives developers all necessary tools for development of applications
for PDA, Pocket PC; and Smartphone.

Here are some advices about different code type choosing.

Use Native Code for achieving the burst performance, direct work with hardware,
and also for minimization of system resource requirements. Use Managed Code for
development of GUI-oriented applications which main requirements are development
and market deployment terms. Managed Code is also good for the easy work with
web-services. Use Server-Side Code for work with different devices through a
single code base, and also if wide stable communication channel with a device is
present.

Below are the reviews of present development tools for PDAs, Pocket PCs and
Smartphones.

eMbedded Visual C++ 4.0

Microsoft eMbedded Visual C++ 4.0 development environment is meant to create
applications for devices with Windows CE .NET 4.2 operating system and also (with
package of updates SP3) for PDA and Smartphones on the Windows Mobile 2003 Second
Edition platform. The development environment eMbedded Visual C++ 4.0 is good for
creation of native code for mobile and inbuilt devices with Windows CE .NET 4.2
OS. It allows developer to perform Just-In-Time Debugging for diagnostics of
unhandled exceptions, Attach-to the process for the extended debugging of
processes, and also interaction with an emulator.

Visual Studio .NET and SDP functions

SDP (Smart Device Programming) functions of the integrated environment of
development Visual Studio .NET 2003 (2005, 2008) allow to create applications,
that uses possibilities of Microsoft .NET Compact Framework platform. Thus a
developer can create the distributed mobile data-processing systems, workings
both in scenarios with connection and without permanent connection. Vast class
library of .NET Compact Framework platform makes application development much
quicker than with traditional development tools.

Visual Studio .NET lets create applications for the Pocket PC 2002 devices and
(with proper SDKs) Pocket PC 2003, 2005 and Smartphone 2003, 2005, applying the
same tools as those used for the development of applications for desktop PCs.
Library .NET Compact Framework is installed together with Visual Studio .NET.
This library is specially developed for devices with the limited resources.
Developers can also use new languages C\# and Visual Basic .NET for applications
for mobile and inbuilt devices. They are good in workings with web-services and
ADO.NET technologies.

Control elements of ASP .NET Mobile Controls extend SDP functions and .NET
Compact Framework platform. They give opportunities to use possibilities of .NET
Compact Framework and Visual Studio .NET for development of mobile
web-applications due to the data delivery to various mobile devices by means of
ASP.NET technology. This approach allows to create single mobile web-application
in the Visual Studio .NET environment, that will automatically perform data
rendering for displaying on various devices: mobile phones, smartphones, PDA,
Pocket PC. The integrated development environment makes it possible to create
mobile web-applications simply by dragging control elements on forms.

ASP.NET System sets no components on a client device. For adaptation of
formatting under concrete browsers server logic is used. It generates information
in the formats of Wireless Markup Language (WML), HTML, and Compact HTML (cHTML).

 2. Building Windows Mobile Applications for Windows Mobile devices using MS
 Visual Studio .NET

Visual Studio development environment together with Compact Framework enable to
develop applications using a vast graphic interface, databases, archiving and
encrypting tools,  USB, Serial, Bluetooth, and also wireless network connections
and telephony. All of these is provided by SDK as well as libraries and heading
files from Platform Builder.

2.1 User Interface

Due to the presence of good designer in VS and SDK we can easily develop GUI for
mobile applications just the same as for Desktop applications. The only
difference is in sets of  graphic components for Desktop systems and for Windows
Mobile. There is also difference between components of Smartphone and Pocket PC
versions caused by that Pocket PC has a touch screen and SmartPhone has not.
Therefore development of interface for SmartPhone is more complex. One should
take it into account when developing software for both platforms.

Here we should also mention that software built for Pocket PC won’t start on the
SmartPhone platform while the opposite situation is possible – till the moment
when some special Smartphone function is called.

It is naturally that many graphic components as well as majority of functions
from full-size Framework were taken away from Windows Mobile SDK to reduce the
size of the SDK on the device side. Only the most needful elements were left. But
the productivity and volumes of memory are increasing so the number of function
included in SDK becomes greater. So the difference between Compact Framework 1.0
and 2.0 is enormous. Version 1.0 was very limited.

In general the development of GUI with Compact Framework on C\# is similar to
the development of the common Win32 application on $C\#$.

It is also possible to develop a graphic interface using assembling of the 3D
rendering-engines. They are such as GAPI (Game API), OPENGL ES (Embedded System),
OPENVG (Vector Graphics), and other projects. Certainly it is rather labor
intensive process as far as it is very important to write an optimal code because
of the relatively low  productivity of mobile devices.

2.2 Communications

To date mobile devices have a wide range of communicational options. They have an
access to the wireless high-speed network using 802.11 WiFi Connection standard.
They also use IrDa, Bluetooth, and USB host/client functionality. While the usage
of Irda is gradually getting less protocols and standards of Bluetooth, WiFi,
Edge, GSM, and also USB are used quit often.

Microsoft Company provides such APIs:

ActiveSync API provides functionality for work with services of synchronization,
file filters, etc. Bluetooth API provides functionality for wireless access to
mobile and peripheral devices. Connection Manager API serves for the automation
of connection process, network connection management. Devices are using
Connection Manager to establish connection and also to inform about the supposed
connection (for example Internet). Object Exchange (OBEX) API provides
functionality for work with effective, compact binary protocol just suitable for
devices with the limited possibilities. Remote API (RAPI) provides functions for
management and remote call of methods on the device side. Such functions are
available: access to the register, files, databases and different configurations
of device from the Desktop-system. The most important option is Remote Procedure
Call when we simply call the method «CeRapiInvoke()» on the Desktop side,
transmit the name of DLL on the device side and the name of function in this DLL
and then just call this method. Pocket Outlook Object Model API provides
functions for work with the objects of Pocket Outlook. It provides interfaces for
synchronization and access to the objects: Task, Calendar, Contacts. Telephony
API (TAPI) includes: Assisted API Extended API Phone API SIM Manager API Short
Messages Service (SMS) API Telephony Service Provider (TSP) API Wireless
Application Protocol (WAP)  API.

Also mobile devices have possibility to work with Serial (USB) ports. There are a
few COM-ports in many devices. Usually the first 3-5 of them are reserved for
IrDa, Bluetooth, SerialPort and others. Other ports are available to for user.
Some devices have USB-Host functionality in other words they are USB On-The-Go
(OTG) devices which can serve both as USB Client and USB Host. For this purpose
device should have necessary Hardware and Software (Device Driver).

Device Driver is a driver which is an intermediate layer between the driver of
HOST and  level of applications. Such driver provides «Stream Interface Driver»
and must contain such functions as:

XXX\_Init XXX\_Open XXX\_Close XXX\_Write XXX\_Read XXX\_IOControl

Here “XXX” is replaced with «prefix» (for example «COM», «DSK»).

This prefix registers in the registry when a driver registers in the system. More
detailed information about Device Driver Interface can be obtained from MSDN.

Also a developer can use such APIs not concerning communicational ones:

Device Configuration API File and Application Management API Game API Home Screen
API HTML Control API MIDI API Shell API Speech Recognizer API Vibrate API Voice
Recorder Control API. 2.3 P/Invoke and Native Interop

As far as Compact Framework contains basic functions and methods from complete

Framework we have such functionality as:

XML Serialization Cryptography Security Reflection Interop Services et al.

Certainly Platform Invoke calls are accessible. For this purpose – as well as in
complete Framework – we use functions and attributes from namespace
System.Runtime.InteropServices and DllImport attribute for description of
functions which will be called from non-managed code.

However the functionality of these methods is limited. So Marshal class has
PtrToStructure, GetComInterfaceForObject, Copy() and Read() functions but, for
example, there is no such function as GetDelegateForFunctionPointer() in it. So
it is impossible to make Marshaling just with Delegate. The
GetFunctionPointerForDelegate function is available only. If we want to transmit
a pointer to the function from Managed code we should get its FunctionPointer and
only after that transmit it to the unmanaged code for subsequent call of it from
there. Also we can not transmit some objects because Compact Framework can not
count SizeOf() for some objects. Therefore frequently we have to transmit  data
of Blitable types only and arrays of these types (Int, byte, char, but not bool),
 and transmit and receive  classes and structures by means of IntPtr. Certainly
it is explained by aspiration to increase the performance. It is therefore
recommended to use primary types for Marshaling and «GCHandle» class for storing
an object in the process memory and to make this memory «visible» for unmanaged
code

Certainly the performance of P/Invoke calls from C\# application in C++ DLL is
almost twice less than the performance of method calls between C++ DLLs. However
the performance of devices grows and it becomes possible to use such calls.

2.4 Debugging

MS Visual Studio enables to make debugging of applications under

Windows Mobile just the same as for Desktop applications. We have Emulator,
Device Emulator Manager and other to functionality such as Breakpoints, Threads,
Watches panel etc.

However to make Debug using Native and Managed code simultaneously is impossible.
Therefore it is possible to start either C++ projects or Managed ones. It’s also
so for «Attach to process». There is a possibility of attaching to the processes
on the device side (or emulator side). ActiveSync is required to provide
communication between a device and the system, it is installed with SDK. An
emulator also can be connected to the computer by means of ActiveSync. We obtain
almost complete emulation of Windows Mobile devices. Here are both SmartPhone and
Pocket PC (or Pocket PC Phone Edition) emulators. However only one Windows Mobile
device can be connected to the computer at one time moment. Therefore Debug on
two devices simultaneously which for example interact in some way is rather
difficult. Certainly we talk only about Debug of applications developed by us but
not about applications and services of the system. For such debugging we need the
complete built of the system created by means of Platform Builder (in the last
versions of Platform Builder it’s included in Visual Studio). We can also create
our own SDK for Visual Studio and Windows CE platform.

An emulator also enables to emulate connection with the GSM network and GPS
support. It makes possible to test and develop large spectrum of applications
without having a physical device in hands.

Talking about Debug with IDA we should mention that version 5.1 already has the
possibility to perform Debug by means of this interactive disassembler with the
use of their plugin which is installed on a device.

With SDK appearance such option as «Deploy» appears in Visual Studio environment.
Now there is «Deploy» item in project context menu additionally to “Build”,
“Clean”, “Debug”. After you choose this item DLL or EXE file built by current
configuration is transmitted to the device. Also new column appears in
Configuration Manager where not only «Build» in some configuration option is
available but also «Deploy». Deployment of .NET projects transmits not only the
unit but also those builds that are depended on it (Dependencies).

3. Deployment and installation

Development of applications for Windows Mobile supposes also their assembling and
creation of installation package. For this purpose there is such concept as
?abinet (.cab) file in Windows Mobile. It is a common archive but it is a
installation package also (a sort of MSI package). It is assembled by means of
Cab Wizard (cabwiz.exe) from MS VisualStudio (or from a command line). This
utility creates processor dependent «.cab» file. WinCEApplicationManager
transmits and starts CAB on a device, so user can install the application using
instructions appearing on device screen. It is also possible to create the
project of MSI Installation which includes CAB file by means of Visual Studio.
This built package will be started on user computer and then transmit ?AB
installation on a device to continue installation process.

There is special Uninstall Manager on the device side. One should attach
configuration INI-file to CAB file. This INI file includes settings to manage
installation process such as path, shortcut name to create, minimal OS version to
install the application and others written in special syntax. It is also possible
to include your own DLL into CAB-file to widen installation options. This DLL
contains such functions as «Installer\_AfterInstall»,
«Installer\_BeforeInstall», «Installer\_AfterUnInstall»,
«Installer\_BeforeUnInstall», code in these functions will be executed on the
device side in the certain moments of installation or uninstallation process.

When developing CAB installation it is possible to attach files, different
resources, built units, add information to the registry. They are designated as
«Project Output. That’s why there is no need to reset anything after CAB-file
creation: it will collect all included projects and files and assemble in an
archive.

4. Security Model for Windows Mobile 5 and Windows Mobile 6

Devices based on Windows Mobile receive, send and analyze potentially important
information which should be protected from unsafe applications. To protect device
the starting of unknown applications is disabled, the access to some API is
restricted and some registry parts changing is forbidden. Units can be marked as
Trusted and Untrusted and system uses this information to prevent starting of
unauthorized applications and limit their access to the system. Also the access
to the system by means of RAPI (Remote API) through ActiveSync can be restricted.
Security Certificates Security Rights are used for the executable units (EXE,
DLL) and for CAB-files.

Protection Against Threats and Risks

The followings options help to protect devices:

Strict password protection. PIN code protection. Devices corrupt deleted
information to prevent access to it (WM 6). Devices corrupt deleted information
on memory cards to prevent access to it (WM 6). Storages encryption and Advanced
Encryption Standard for SSL (WM 6). Own certificates usage. Detecting a device
via Bluetooth can be protected (WM 6 Smartphone). 2 layer system of application
starting (One-tier and Two-tier access). No support of macros, therefore viruses
can make much harm.

The start of applications is based on Permissions. Windows Mobile devices use
such models:

Privileged Normal Blocked

Privileged applications have the widest access. They can access any API, write in
the protected areas of registry and have the complete access to the system.

Most applications have Normal model. They do not have access to the trusted API
and do not have the complete control of the system.

Application can not be started at all if it has “Blocked” status. It means that
it’s not signed with the proper certificate or user forbade the start by the
proper warning of the system.

Analogical situation is with Cab-files. The executable units with Normal status
can start Privileged DLL but then they will work as Normal. However Privileged
executable units can not start the units having Normal status.

Written by Eugene Kordin, Apriorit specialist.
