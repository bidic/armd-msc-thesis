Przedmiotem pracy magisterskiej jest rozwój systemu robota mobilnego zrealizowanego pierwotnie w pracy magisterskiej pt.: ,,Rozwój systemu sterowania dla robota mobilnego''\cite{KmakMScThesis2009}. Temat pracy daje dużą swobodę w wyborze kierunku rozwoju części sprzętowej oraz programowej robota. Niniejsza praca opisuje proces rozbudowy robota Dark Explorer, począwszy od analizy konfiguracji początkowej, poprzez poprawę wydajności istniejących elementów, aż po dodanie całkiem nowych funkcjonalności. Praca magisterska wymagała wiedzy z zakresu elektroniki, fizyki oraz szeroko pojętej informatyki. 

W ramach pracy magisterskiej zostały podłączone oraz oprogramowane czujniki składające się na inercjalną jednostkę pomiarową. Jednostka ta umożliwia rejestrowanie toru ruchu, po którym robot jest przenoszony przez użytkownika. Podjęto również próbę stworzenia inercjalnego systemu nawigacyjnego w celu rozwiązania problemu rekonstrukcji ścieżki powrotnej na podstawie danych zarejestrowanych przy pomocy czujnika przyspieszenia oraz żyroskopu. W trakcie odtwarzania powrotnego toru ruchu robot, wykorzystując dane z dołączonych do niego dalmierzy, jest w stanie uniknąć zderzenia z napotkanymi przeszkodami jednocześnie starając się dotrzeć do punktu startowego. W pracy magisterskiej zawarty jest również przewodnik pozwalający programistom, nie posiadającym doświadczenia w tworzeniu oprogramowania dla systemów wbudowanych, zaznajomić się z narzędziami pomocnymi w rozwoju programów sterujących mikrokontrolerami opartymi o architekturę ARM. Poruszono także tematykę rozwoju aplikacji sterujących robotem dedykowanych na urządzenia mobilne oraz stacjonarne. Efektem tego jest stworzone oprogramowanie działające na komputerach oraz telefonach z obsługą technologii Java ME oraz .NET. Podczas tworzenia oprogramowania położono szczególny nacisk na przenośność wszystkich rozwiązań na różne środowiska uruchomieniowe oraz intuicyjną obsługę i prostą rozbudowę oprogramowania.

W rozdziale pierwszym przedstawiono dzieje robotyki od jej początków, aż po dzień dzisiejszy. W ramach tego rozdziału, zaprezentowane zostały dziedziny w jakich współcześnie roboty znajdują zastosowanie. Nie zabrakło również podstawowych informacji na temat sposobów klasyfikacji i podziału robotów. Rozdział ten ma na celu wprowadzenie użytkownika w zagadnienia z którymi robotyka zmaga się na co dzień.

Rozdział drugi przedstawia analizę sprzętu oraz oprogramowania robota Dark Explorer stworzonych w ramach poprzedniej pracy magisterskiej. Zwraca on uwagę na możliwości oraz potencjalne przeszkody które mogą się pojawić podczas rozwoju robota. 

Trzeci rozdział opisuje narzędzia, oraz sposób ich konfiguracji, niezbędne podczas rozwoju oprogramowania dla systemów wbudowanych. Ze względu na znaczące różnice pomiędzy narzędziami przeznaczonymi dla systemu Windows i Linux, zostały one omówione w ramach oddzielnych podrozdziałów. 

W rozdziale czwartym poruszana jest tematyka związana z tworzeniem oprogramowania sterującego robotem działającego na urządzeniach mobilnych oraz stacjonarnych. 

Rozdział piąty przedstawia rozszerzenia warstwy sprzętowej wprowadzone w trakcie realizacji pracy magisterskiej. Czytelnik znajdzie tutaj szczegółowy opis zasady działania i możliwych zastosowań dla czujników dołączonych do nowej wersji robota Dark Explorer. W ramach rozdziału omówiono również sposób konstrukcji obudowy pozwalającej na wygodne podłączanie wszystkich elementów rozszerzających dotychczasowe możliwości robota. 

Szósty rozdział prezentuje funkcje oprogramowania systemu wbudowanego rozwiązujące problemy takie jak: rekonstrukcja ścieżki powrotnej, omijanie przeszkód, pobieranie obrazu z kamery z maksymalną dostępną rozdzielczością oraz detekcja twarzy na obrazie statycznym. Opisany jest również protokół komunikacji bluetooth warstwy aplikacji modelu referencyjnego ISO.

Złożoność problemów realizowanych w ramach tej pracy magisterskiej oraz szeroki zakres wiedzy konieczny do jej realizacji wymagał aby była ona realizowana w zespole dwuosobowym. Szczegółowy opis z podziałem wkładu każdego z autorów dostępny jest na końcu tej pracy.