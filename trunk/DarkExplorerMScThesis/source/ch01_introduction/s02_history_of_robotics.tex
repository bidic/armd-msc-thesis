\chapter{Historia i rozwój robotyki}
Robotyka jest stosunkowo młodą dziedziną łączącą różne gałęzie nauk technicznych
i nie tylko. Do pełnego zrozumienia zagadnień współczesnej robotyki oraz budowy
i~zastosowań robotów konieczna jest niejednokrotnie rozległa wiedza na
temat elektroniki, mechaniki, inżynierii przemysłowej, matematyki oraz szeroko pojętej
informatyki. Ponadto wiele nowopowstałych gałęzi wiedzy zajmujących się rozwojem
sztucznej inteligencji, modelowaniem sztucznego życia czy rozwojem inżynierii
wiedzy coraz częściej staje się nierozerwalnie związane z problemami współczesnej
robotyki. Nie~należy zapominać również o tym, że rozwój inżynierii wytwarzania
oraz automatyki przemysłowej pozwala na nieustanny rozwój robotyki z
którą mamy do czynienia w przemyśle w dniu dzisiejszym.

\section{Historia robotyki}
Pojęcie ,,robot'' w literaturze pojawiło się po raz pierwszy w sztuce czeskiego
pisarza Karel'a Capka w roku 1920. Termin ,,robot'' oznacza w języku czeskim
pracę lub służbę przymusową. Nieco ponad 20 lat później, amerykański uczony i
pisarz Isaac Assimov w jednym ze swoich opowiadań po raz pierwszy używa słowa
robotyka. W kolejnych latach Assimov w swojej twórczości niejednokrotnie wraca do
problemu robotyki skutkiem czego jest wydanie w 1950 roku zbioru opowiadań pod
tytułem ,,Ja, robot''. Jako ciekawostkę można dodać, że w wydanym w 1942 roku
opowiadaniu pod tytułem ,,Zabawa w berka'', Assimov wprowadza trzy prawa
robotyki według których, zdaniem autora, powinny być programowane
roboty\cite{Runaround}. Zdefiniowane przez Assimov'a prawa robotyki w pełnym
brzmieniu widoczne są na diagramie \ref{fig:Assimov_Laws}.
% \begin{description}
% \item[Prawo pierwsze] \hfill \\
% Robot nie może zranić istoty ludzkiej, ani przez zaniedbanie narazić człowieka
% na zranienie. 
% \item[Prawo drugie] \hfill \\
% Robot musi spełniać polecenia wydawane przez człowieka, poza poleceniami
% sprzecznymi z prawem pierwszym.
% \item[Prawo trzecie] \hfill \\
% Robot musi chronić samego siebie dopóki nie jest to sprzeczne z prawami
% pierwszym i drugim.
% \end{description} 

\begin{figure}[h!]
	\centering
	\includegraphics[height=70mm]{../images/ch01/assimov_laws.png}
	\caption{Prawa robotyki zdefiniowane przez Issaca Assimov'a}
	\label{fig:Assimov_Laws}
\end{figure}

Nieco później, Isaac Assimov, jako uzupełnienie i prawo nadrzędne dodaje prawo
zerowe o następującym brzmieniu ,,Robot nie może szkodzić ludzkości, ani przez
zaniedbanie narazić ludzkości na szkodę''\cite{website:robotyka-pl}. Oprócz
wspomnianych powyżej podstawowych praw, zdefiniowano również inne prawa
wynikające z prowadzonych w tej dziedzinie badań i rozwoju robotyki.
\newpage
Takim sposobem urządzenie mechaniczne wykonujące zadania w sposób automatyczny
otrzymało miano ,,robota''\cite{website:asimo-pl}. Operacje wykonywane przez
robota mogą być bezpośrednio kontrolowane przez człowieka na podstawie przygotowanego wcześniej
programu zawierającego zestaw reguł które umożliwiają robotowi wykonywanie
określonych czynności. Możliwość wyręczenia człowieka przez maszynę w
wykonywaniu monotonnych, złożonych i powtarzalnych czynności, niejednoktornie
z dużo większą wydajnością i precyzją, była jednym z podstawowych bodźców
który sprzyjał rozwojowi robotyki już od samego jej początku. Pojęcie robot,
używane jest również w stosunku do autonomicznie działających urządzeń
pobierających informację z~otocznia za pomocą różnego rodzaju czujników,
nazywanych sensorami, oraz oddziałujących na swoje otoczenie i reagujące na
jego zmiany. 

Dzięki gwałtownemu rozwojowi nauki i techniki w czasie II wojny światowej w roku
1956 GC.~Devol i JF. Engelberger zainspirowani twórczością Assimov'a zaprojektowali i
stworzyli pierwszy w dziejach ludzkości działający egzemplarz robota\cite{website:robotyka-pl}.
Engelberger założył firmę pod nazwą ,,Unimation'' zajmującą się automatyzacją,
pierwszym robotem stworzonym przez firmę Engelbergera był robot nazwany
,,Unimate''. Został on zainstalowany w fabryce General Motors w Trenton gdzie
został przystosowany do obsługi wysokociśnieniowej maszyny odlewniczej.
W~kolejnych latach roboty firmy ,,Unimation'' znalazły swoje zastosowanie w
innych gałęziach przemysłu, a sam Engelberger otrzymał miano ojca
robotyki.\cite{website:robotyka-pl}

Nieco później, bo w roku 1979 Robotics Industries Association 
zdefiniowało robota jako wielofunkcyjny, programowalny manipulator
zaprojektowany do przenoszenia materiałów, narzędzi, części urządzeń poprzez
serię programowalnych ruchów wykonywanych w celu realizacji różnych zadań.
W myśl wspomnianej definicji jedną z podstawowych cech robota jest jego
programowalność i możliwość dostosowywania się do zmiennych warunków środowiska
pracy. 

Pierwsze roboty projektowano z myślą o zastosowaniu ich do realizacji
elementarnych zadań związanych z przenoszeniem elementów z jednego punktu do
drugiego. Program pracy robota miał więc charakter sekwencji operacji
prowadzących do realizacji określonego przez programistę zadania. W miarę
rozwoju technicznego, stawiane przed robotami zadania wymusiły użycie przez
konstruktorów czujników które pozwalały na zwiększenie poziomu interakcji
robotów z otoczeniem, a co za tym idzie umożliwiły realizowanie zadań o wysokim
stopniu złożoności. Od tego momentu kierunek rozwoju robotyki obrał
sobie za cel stworzenie maszyny na tyle uniwersalnej i niezależnej aby mogła
nosić miano androida. Jednak do realizacji tego zadania konieczne jest
opracowanie sztucznej inteligencji na którą z pewnością przyjdzie nam jeszcze
trochę poczekać. 

Pierwszym krokiem na drodze do stworzenia androida było oderwanie robota
od~stałego miejsca instalacji i pozwolenie mu w miarę możliwości swobodne
poruszanie się w~dostępnej mu przestrzeni roboczej. Tak powstała klasa robotów
które mogą przemieszczać się za pomocą kół, gąsienic czy nawet kończyn lub
odnóży. Roboty tego typu potrafią pływać, latać i sprawnie poruszać się po lądzie
dodatkowym ich atutem jest fakt iż większość z nich posiada niemal całkowitą
autonomię i ograniczona jest jedynie poprzez wielkość otoczenia w jakim zostały
umieszczone. Takim sposobem powstała grupa robotów nazywanych robotami mobilnymi.
Cechą wspólną wszystkich urządzeń z~tej rodziny była umiejętność swobodnego
przemieszczania się oraz analiza najbliższego otoczenia. Na~tej podstawie maszyna
mogła przeprowadzić wnioskowanie pozwalające na~podejmowanie dalszych akcji czy
też przesłanie użytkownikowi odczytanych parametrów środowiska.

Historię robotów mobilnych zapoczątkował amerykański Uniwersytet w Stanford gdyż
w roku 1968 jako pierwszy stworzył w pełni działający model robota mobilnego pod
nazwą Shakey. Nazwa została zainspirowana szarpanymi ruchami z jakimi robot się
poruszał. Głównym zadaniem robota było modelowanie otoczenia w którym się
znajdował. W ślad za Uniwersytetem w Stanford ruszyło MIT. W roku 1983 posiadali
już pierwszy działający model robota swobodnie skaczącego. Niecałe 6
lat później MIT stworzyło pierwszego robota kroczącego wzorowanego na owadach.
Robot ten sterowany był za pomocą wielowarstwowych automatów o stanach
skończonych, a ze światem zewnętrznym komunikował się za pomocą czułek,
inklinometrów, czujników zbliżeniowych na podczerwień. Genghis, bo tak
nazwany został robot, posiadał na pokładzie 4 ośmiobitowe jednostki obliczeniowe, ważył
niecały kilogram i miał 35 cm długości.

\begin{figure}[h!]
 \centering
 \includegraphics[height=50mm]{../images/ch01/shakey_and_genghis.png}
 \caption{Od lewej: Shakey (Stanford), Genghis (MIT) }
 \label{fig:RobotsHistory_Shakey_Genghis}
\end{figure}

Po sukcesie wspomnianych projektów, rozwój robotów mobilnych następował już
bardzo dynamiczne. W latach 90 powstawało wiele różnych modeli robotów o bardzo
różnorodnych rodzajach napędów oraz zestawach czujników umożliwiających
interakcje ze światem zewnętrznym.

Swoistym ukoronowaniem prac było w 1997 roku stworzenie przez NASA robota o
nazwie Pathfinder. Robot wyposażony był w czujniki laserowe, stereowizję,
żyroskopy i inne rodzaje czujników o charakterze badawczym. Zasilany był on
bateriami słonecznymi które pozwoliły mu na 83 dni nieprzerwanej pracy podczas
której robot przebył około 100 metrów i wykonał 230 manewrów. W ostatnich
latach do największych osiągnięć robotyki mobilnej można z pewnością zaliczyć
powstanie robotów humanoidalnych takich jak japoński ASIMO. 

\begin{figure}[h!]
 \centering
 \includegraphics[height=60mm]{../images/ch01/pathfinder_and_asimo.png}
 \caption{Kolejno: Pathfinder (NASA), Asimo (Honda)}
 \label{fig:RobotsHistory_Pathfinder_Asimo}
\end{figure}

Robot ten ważył 54 kg i posiadał 130 cm wysokości. Wersja z roku 2005 potrafiła
biegnąc osiągnąć prędkość dochodzącą nawet do do 6 km/h. Ponadto robot potrafił
wchodzić w interakcję z otaczającymi go ludźmi i przedmiotami. Urządzenie
stworzone przez inżynierów z firmy Honda potrafiło rozpoznawać gesty takie jak
podanie ręki, wskazanie kierunku czy machanie ręką na pożegnanie. Robot równie
dobrze radził sobie z rozpoznawaniem twarzy, dźwięków i analizą otaczającego go
środowiska. Potrafił on rozpoznać i omijać niebezpieczeństwa postawione na jego
drodze jak na przykład schody czy osoby poruszające się w jego kierunku.

W roku 2006 podczas Robot World w Seulu grupa południowokoreańskich inżynierów
prezentuje swojego robota EveR-2. EveR są robotami posiadającymi wygląd typowej
dwudziestoletniej koreanki. Urządzenie potrafi rozmawiać i śpiewać dzięki
wbudowanemu ,,silnikowi dialogu'' (ang. embedded dialogue engine). Android
EveR-2 w odróżnieniu od swojej poprzedniczki ma udoskonalony system wizyjny
oraz możliwość wyrażania emocji takich jak znudzenie, zadowolenie, żal czy radość.
Robot ma 170cm wzrostu i waży około 60kg. Twarz androida posiada wysoką
elastyczność i poruszana jest przy pomocy 29 silniczków i licznych stawów
które pozwalają na pełną swobodę w wyrażaniu emocji. Wbudowany system
rozpoznawania i syntezy mowy w połączeniu z możliwością wyrażania się za pomocą
gestów pozwala na niemal całkowitą swobodę podczas rozmowy z robotem. Możliwości
komunikacyjne robota są tak znakomite, że EveR-2 została pierwszą piosenkarką
androidem. Jej pierwszy występ odbył się podczas jej prezentacji w Seulu gdzie
na oczach publiczności odśpiewała koreańską balladę pt. ,,I will close my eyes
for you''.

Lata 2008 i 2009 zaowocowały powstaniem wielu robotów naśladujących w swoim
zachowaniu niektóre zwierzęta. Bardzo imponującym przykładem takiego robota jest
wyprodukowany przez firmę AeroVironment latający robot o nazwie Mercury.
Urządzenie to potrafi ,,zawisnąć'' w powietrzu poruszając jedynie skrzydłami
dokładnie w taki sam sposób jak robią to prawdziwe kolibry. Lata te były również
obfite w sukcesy w dziedzinie rozwoju sztucznego mózgu i sztucznej inteligencji.
W roku 2008 grupa naukowców z University of Reading zbudowała robota sterowanego w całości za
pomocą biologicznego mózgu z wyhodowanych neuronów. Biologiczny mózg w który wyposażony
został zainstalowany w macierzy wieloelektrodowej (MEA). MEA jest to pewnego
rodzaju naczynie które za pomocą około 60 elektrod przechwytuje sygnały
elektryczne generowane przez komórki nerwowe i przekłada je na ruchy robota. W
przypadku gdy robot napotka na swojej drodze przeszkody wspomniane
wcześniej elektrody stymulują komórki nerwowe które tą samą drogą udzielają
instrukcji na zachowania kół które pozwoliłoby na ominięcie wykrytej przeszkody. Robot jest
całkowicie samodzielny i w całości sterowany przez własny mózg.\\
\\
Obserwując postęp w dziedzinie robotyki można odnieść wrażenie iż w dzisiejszych
czasach roboty znalazły dla siebie zastosowanie niemal w każdej dziedzinie życia.
Od wielu lat sprawdzają się już w przemyśle, transporcie, budownictwie oraz są
niezastąpione w środowiskach nieprzyjaznych człowiekowi, takich jak podmorskie
głębiny czy otchłań kosmosu. Obszar zastosowań robotów jest tak szeroki iż
wydawać się może, że jedynym czynnikiem ograniczającym rozwój współczesnej
robotyki są względy czysto technologiczne. Nie staje to jednak na przeszkodzie do
projektowania i tworzenia przez konstruktorów z całego świata rozwiązań coraz
bardziej przybliżających ludzkość do stworzenia w pełni samodzielnego oraz
inteligentnego androida.
\section{Obecny rozwój i zakres zastosowań robotyki}
Robotyka zawładnęła wieloma dziedzinami życia człowieka od przemysłu poprzez
zastosowania medyczne, wojskowe aż po urządzenia stosowane w gospodarstwach
domowych. Roboty znalazły dla siebie zastosowanie w wykonywaniu zadań
wymagających dużej szybkości, dokładności i wytrzymałości której nie może
zapewnić praca wykonywana przez ludzi. W rezultacie wiele zadań wykonywanych w
produkcyjnych zakładach pracy, dawniej wykonywane przez ludzi, zostało zastąpione
przez roboty. Efektem tego jest zmniejszenie kosztów produkcji towarów masowych,
szczególnie widocznych w przypadku części samochodowych i elektroniki.
Zastosowanie robotów powoduje więc wzrost efektywności ekonomicznej i skraca czas
uruchomienia produkcji, a co za tym idzie jest głównym czynnikiem ekonomicznym wspomagającym rozwój robotyki.

Istnieje szereg zadań które człowiek wykonuje lepiej niż maszyna,
ale ze względu na męczący charakter pracy lub niebezpieczne środowisko jej wykonywania praca
ludzka jest zastępowana przez maszyny. Stały rozwój technologii stosowanych w
robotyce skutkuje w powstawaniu coraz bardziej zaawansowanych systemów co sprzyja
powszechniejszemu ich stosowaniu. Zastosowanie robotów w życiu codziennym
prowadzi do zwiększenia bezpieczeństwa pracy szczególnie na stanowiskach pracy
zagrażających zdrowiu i życiu człowieka. Co więcej stały rozwój robotyki pozwala
na prowadzenie badań w lokalizacjach fizycznie niedostępnych dla człowieka.
Bardzo dobrym tego przykładem jest eksploracja odległych planet czy też wnętrz
wulkanów. Jak można więc zauważyć rozwój robotyki stał się w chwili obecnej
samonapędzającą się maszyną, w której powstawanie coraz doskonalszych robotów
zwiększa na nie zapotrzebowanie, a to z kolei wymusza dalszy ich rozwój.
\newpage\section{Klasyfikacja robotów}
Współczesna różnorodność robotów doprowadziła do powstania wielu podziałów
robotów ze względu na szereg różnych parametrów. Jednym z bardzo często
spotykanych podziałów jest klasyfikacja ze względy na sposób programowania i
możliwości komunikacyjne. Każda z grup tworzy tzw. generacje robotów. Można
wyróżnić trzy generacje robotów uwzględniając różnice w ich układzie sterowania
oraz dostępne sensory\cite{website:robotyka-pl}.
% \begin{description}
% \item[Roboty I generacji] to urządzenia zaprogramowane do wykonywania określonej
% sekwencji czynności z możliwością ich przeprogramowania. W robotach tej
% generacji stosuje się otwarty układ sterowania który charakteryzuje się
% całkowitym brakiem możliwości pobierania informacji ze świata zewnętrznego. Do
% robotów pierwszej generacji zaliczyć można roboty przemysłowe przeznaczone do
% podawania i odbierania obiektów z linii produkcyjnej.
% \item[Roboty II generacji] to maszyny wyposażone w zamknięty układ sterowania.
% Oznacza to, że posiadają one zestaw czujników umożliwiających dokonywanie
% pomiarów stanu robota oraz jego otoczenia. Pozwala to robotowi na rozpoznanie
% żądanego obiektu nawet wówczas, gdy przemieszcza się wśród innych obiektów.
% Robot tej generacji powinien być w stanie podjąć decyzję na temat sposobu
% realizacji zadania na podstawie aktualnego stanu otoczenia i obiektu
% manipulacji.
% \item[Roboty III generacji] to urządzenia również wyposażone w zamknięty układ
% sterowania oraz zestaw czujników umożliwiających dokonywanie skomplikowanych
% pomiarów i klasyfikację obiektów o wysokim stopniu złożoności. System sterowania
% robota tej generacji powinien pozwalać na jego adaptację w nieznanym otoczeniu.
% \end{description}
\begin{figure}[h!]
 \centering
 \includegraphics[height=120mm]{../images/ch01/robot_generations.png}
 \caption{Klasyfikacja robotów ze względu na ich generację}
 \label{fig:RobotsGenerations}
\end{figure}

Istnieje szereg różnych klasyfikacji robotów bazujących na podziale ze względu
na ich parametry techniczne. Jednym z bardziej znanych podziałów jest
przedstawiona poniżej klasyfikacja ze względu na rodzaj środowiska w jakim się
poruszają i sposób ich poruszania się.
\begin{itemize}
  \item roboty podwodne i poruszające się na wodzie,
  \item roboty lądowe oraz wodno-lądowe
  \item roboty powietrzne,
\end{itemize}

Wśród robotów lądowych bardzo popularną grupą są roboty mobilne, podział których
można przeprowadzić na podstawie sposobie poruszania się. Za pomocą takiego
kryterium możemy wyróżnić roboty kołowe, kroczące, skaczące oraz pełzające.
Innym popularnym sposobem klasyfikacji robotów jest podział ze względu na 
obszar ich zastosowania. Stosując kryterium takiego rodzaju istnieje możliwość
wyróżnienia następujących grup:
% \begin{itemize}
%   \item roboty przemysłowe,
%   \item roboty badawczo-rozwojowe, eksploracyjne, poszukiwacze, kosmiczne,
%   \item roboty wojskowe i policyjne,
%   \item roboty do gospodarstwa domowego,
%   \item roboty usługowe sektora publicznego (społeczne, interaktywne i
%   terapeutyczne),
%   \item roboty medyczne i egzoszkielety,
% \end{itemize}
\begin{figure}[hb]
 \centering
 \includegraphics[height=70mm]{../images/ch01/robot_types.png}
 \caption{Podział robotów ze względu na obszar ich zastosowania}
 \label{fig:RobotsDiv}
\end{figure}

Omówione sposoby klasyfikacji nie wyczerpują w pełni problemu podziału
robotów i ich zastosowań. Istnieje bowiem wiele charakterystyk opartych o takie
parametry jak na przykład kształt czy rodzaj układu napędowego o których w tym
rozdziale nie wspomniano. Celem przedstawionych informacji było zakreślenie
obszaru zastosowań robotów oraz bogactwa ich różnorodności, a co za tym idzie
uświadomienie czytelnikowi obszarów zastosowań robotyki we współczesnym świecie
jak również potencjalnych problemów z jakim na codzień zmagają się ludzie
projektujący i budujący roboty.
