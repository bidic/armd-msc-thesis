\section{Lokalizcja twarzy na obrazie}

W ciągu ostatnich lat obserwuje się bardzo dynamiczny rozwój biometrii.
Aplikacje analizujące biometryczne cechy użytkowników znalazły swoje
zastosowanie miedzy innymi w systemach zabezpieczeń i monitoringu, robotyce,
kopmuterach przenośnych, aparatach i kamerach cyfrowych czy nawet w nowoczesnych
telefonach komórkowych. Pośród szeregu cech które mogą zostać poddane analizie
szczególnym zainteresowaniem cieszy się lokalizacja ludzkiej twarzy. Systemy
posiadające możliwość zlokalizowania na obrazie twarzy od pewnego czasu
towarzyszą nam w życiu codziennym. Niemniej jednak większość istniejących
algorytmów pozwalających na rozwiązanie problemu lokalizacji twarzy jest dosyć
skomplikowana i ma bardzo konkretne wymagania zarówno co do platformy sprzętowej
jak i jakości obrazu który będzie poddawany analizie. Dlatego też stworzenie
implementacji dla platformy o bardzo ograniczonych zasobach obliczeniowych i
pamięciowych jest nielada wyzwaniem. Celem niniejszego rozdziału jest
przedstawienie podstawowych rodzajów algorytmów lokalizacji twarzy wskazanie ich
wad oraz zalet. W kolejnym podrodziale znajduje się szczegółowy opis stworzonej
implementacji algorytmu zastosowanego w realizowanej pracy magisterskiej.


Applications based on human face detection have been
significantly developed recently—surveillance systems,
digital monitoring, intelligent robots, notebook, PC cameras,
digital cameras, 3G cell phones, and the like. These
applications consequently play an important role in our daily
life. Nevertheless, the algorithms of the applications are quite
complicated and hard to meet real-time requirements of
specific frame-rate.
Over the past decade, many approaches for improving the
performance of human face detection have been proposed,
which are categorized into to main types (1) Knowledge-based
method: This method is aimed at finding invariant features of
a face within a complex environment, thereby localizing the
position of the face. Relationships among the features
helpfully determine whether a human face appears in an image
or not [1]. (2)Feature invariant approaches: Invariant features,
unresponsive to different positions, brightness, and viewpoints,
are utilized in this approach to detect human faces. A
statistical model is usually built up for describing the relations
among face features and the presence of the detected faces.
Such face features, for instance, are Facial Features [2],
Texture [3], and Skin Color [4]. (3)Template matching
method: A template cohering with human face features is used
to perform a pattern-matching operation based on the template
and an input image. Shape template [5] and Active Shape
Model [6] are common examples of this method.
(4)Appearance-based method: This method, such as Eigen
face [7], Neural Network [8], and Hidden Markov Model [9],
employs a series of face images to train and establish a face
model for the face detection. In general, method (2), (3), and
(4) are more complex than method (1); yet the more features
are used in method (1), the more complicated it is.
In this paper, a complexity-reduced algorithm for detecting
human faces in real-time is proposed. This algorithm, on the
strength of Knowledge-based method with minimum features,
adopts the geometric characteristics of skin and hair color to
detect human faces. In addition, this algorithm is able to be
expectedly transplanted to an embedded system, like the
developing pet robot
\subsection{Algorytmy}
\subsection{Implementacja}
Implementacja modułu do lokalizacji twarzy na obrazie pobranym z kamery
robota została w całości oparata o algorytm zaproponowany przez Yao-Jiunn Chen,
Yen-Chun Lin w artykule zatytułowanym 'Simple Face-detection Algorithm Based on
Minimum Facial Features'. 
Pierwszym krokiem na drodze do zlokalizowania twarzy na obrazie jest akwizycja
danych z kamery. Przesłany z kamery obraz automatycznie jest poddawany
podstawowej korekcji mającej na celu znormalizownie balansu bieli oraz
dopasowanie ostrości. Tak przygotowany obraz poddawany jest binaryzacji. Celem
wspomnianego procesu jest odnalezienie na obrazie obszarów będących w kolorze
skóry oraz 
